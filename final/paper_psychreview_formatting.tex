\documentclass[11pt,letterpaper]{article}

%\usepackage{showlabels}
\usepackage{fullpage}
\usepackage{pslatex}
%\usepackage{latexsym}
\usepackage[english]{babel}
\usepackage[utf8]{inputenc}
\usepackage{amsmath}
\usepackage{bm}
\usepackage{graphicx}
\usepackage{tikz}
\usepackage{xcolor}
\usepackage{url}
%\usepackage[colorinlistoftodos]{todonotes}
\usepackage{rotating}
\usepackage{natbib}
\usepackage{amssymb}
\usepackage{lingmacros}

\usepackage{tikz-dependency}
\usepackage{longtable}

\usepackage{changepage}


%\usepackage{paralist} 
%\usepackage{graphicx} 
%\usepackage{multirow} 
%\usepackage{enumitem}
\usepackage{linguex}
%\raggedbottom

\definecolor{Purple}{RGB}{255,10,140}
\newcommand{\jd}[1]{\textcolor{Purple}{[jd: #1]}} 


\newcommand{\R}[0]{\mathbb{R}}
\newcommand{\E}[0]{\mathbb{E}}
\newcommand{\Ff}[0]{\mathcal{F}}

\usepackage{multirow}

\newcommand{\soft}[1]{}
\newcommand{\nopreview}[1]{}
\newcommand\comment[1]{{\color{red}#1}}
\newcommand\mhahn[1]{{\color{red}(mhahn #1)}}
\newcommand\jdegen[1]{{\color{red}(jdegen #1)}}
\newcommand\note[1]{{\color{red}(#1)}}
\newcommand\REF[0]{{\color{red}(REF) }}
\newcommand\CITE[0]{{\color{red}(CITE) }}
\newcommand\rljf[1]{{\color{red}(#1)}}
\newcommand{\key}[1]{\textbf{#1}}
\newcommand\revision[1]{{#1}}
\newcommand\revdeleted[1]{{}}


%\usepackage[dvipsnames]{xcolor}

\usepackage{amsthm}

\newcommand{\thetad}[0]{{\theta_d}}
\newcommand{\thetal}[0]{{\theta_{LM}}}

\newcounter{theorem}
\newcounter{def}
\newtheorem{proposition}[theorem]{Proposition}
\newtheorem{thm}[theorem]{Theorem}
\newtheorem{definition}[def]{Definition}
\newtheorem{corollary}[theorem]{Corollary}
\newtheorem{question}[theorem]{Question}
\newtheorem{example}[theorem]{Example}
\newtheorem{lemma}[theorem]{Lemma}


\frenchspacing
%\def\baselinestretch{0.975}

%\emnlpfinalcopy
%\def\emnlppaperid{496}

\title{Modeling word and morpheme order in natural language as an efficient tradeoff of memory and surprisal}
\author{\begin{tabular}{ccc}Michael Hahn & Judith Degen & Richard Futrell \\ Stanford University & Stanford University & UC Irvine \\ mhahn2@stanford.edu & jdegen@stanford.edu & rfutrell@uci.edu \end{tabular}}
\date{}

\begin{document}

\maketitle


\begin{abstract}
Memory limitations are known to constrain language comprehension and production, and have been argued to account for crosslinguistic word order regularities.
However, a systematic assessment of the role of memory limitations in language structure has proven elusive, in part because it is hard to extract precise large-scale quantitative generalizations about language from existing mechanistic models of memory use in sentence processing.
We provide an architecture-independent information-theoretic formalization of memory limitations which enables a simple calculation of the memory efficiency of languages.
Our notion of memory efficiency is based on the idea of a \emph{memory--surprisal tradeoff}: a certain level of average surprisal per word can only be achieved at the cost of storing some amount of information about past context.
Based on this notion of memory usage, we advance the \emph{Efficient Tradeoff Hypothesis}: the order of elements in natural language is under pressure to enable favorable memory-surprisal tradeoffs.
We derive that languages enable more efficient tradeoffs when they exhibit \emph{information locality}: when predictive information about an element is concentrated in its recent past.
We provide empirical evidence from three test domains in support of the Efficient Tradeoff Hypothesis:
a reanalysis of a miniature artificial language learning experiment, a large-scale study of word order in corpora of 54 languages, and an analysis of morpheme order in two agglutinative languages. These results suggest that principles of order in natural language can be explained via highly generic cognitively motivated principles and lend support to efficiency-based models of the structure of human language.
\end{abstract}

%\jd{2 stylistic notes on plots: 1. it would generally be better to use a color-blind friendly palette. 2. it would be helpful if a particular color had a particular semantics throughout the paper, eg, if real languages across plots always had the same color, if different baselines that correspond to analogous versions of each other across plots did, etc. not strictly necessary, just makes it a lot easier on the reader.}
 
\section{Introduction}
\footnotetext{© 2021, American Psychological Association. This article may not exactly replicate the authoritative document published in the APA journal. It is not the copy of record. The copy of record is available at \url{https://doi.apa.org/doi/10.1037/rev0000269} \\ All code and data are freely available at \url{https://github.com/m-hahn/memory-surprisal} . The SI appendix is available at \url{https://doi.org/10.1037/rev0000269.supp} .}


Natural language is a powerful tool that allows humans to communicate, albeit under inherent cognitive resource limitations.
Here, we investigate whether human languages are grammatically structured in a way that reduces the cognitive resource requirements for comprehension, compared to counterfactual languages that differ in grammatical structure.

The suggestion that the structure of human language reflects a need for efficient processing under resource limitations has been present in the linguistics and cognitive science literature for decades~\citep{yngve1960model,berwick1984grammatical,hawkins1994performance,chomsky2005three,jaeger2011language,gibson2019efficiency,hahn2020universals}. The idea has been summed up in \citeauthor{hawkins2004efficiency}'s (\citeyear{hawkins2004efficiency}) \key{Performance--Grammar Correspondence Hypothesis} (PGCH), which holds that grammars are structured so that the typical utterance is easy to produce and comprehend under performance constraints.

One major source of resource limitation in language processing is incremental memory use. 
When producing and comprehending language in real time, a language user must keep track of what they have already produced or heard in some kind of incremental memory store, which is subject to resource constraints.
These memory constraints have been argued to underlie various \key{locality principles} which linguists have used to predict the orders of words within sentences and morphemes within words \citep[e.g.][]{behaghel1932deutsche,givon1985iconicity,bybee-morphology-1985,rijkhoff-explaining-1990,hawkins1994performance,hawkins2004efficiency,hawkins2014crosslinguistic,temperley-minimizing-2018}.
The idea is that language should be structured to reduce long-term dependencies of various kinds, by placing elements that depend on each other close to each other in linear order.
That is, elements of utterances which are more `relevant' or `mentally connected' to each other are closer to each other.
%For example, \citet{hawkins-efficiency-2003,hawkins2014crosslinguistic} has used a locality principle, `domain minimization,' to explain cross-linguistic universals of word order \citep[see also][for reviews of this idea and its predictions]{liu-dependency-2017,temperley-minimizing-2018}.

Our contribution is to present a new, highly general formalization of the relationship between sequential order and incremental memory in language processing, from which we can derive a precise and empirically testable version of the idea that utterance elements which depend on each other should be close to each other. 
%We do so by introducing a new information-theoretic formalization of the relationship between sequential order and memory usage in language processing. 
Our formalization allows us to predict the order of words within sentences, and morphemes within words directly by the minimization of memory usage.

We formalize the notion of memory constraints in terms of what we call the \key{memory--surprisal tradeoff}: the idea that the ease of comprehension depends on the amount of computational resources invested into remembering previous linguistic elements, e.g., words. 
Therefore, there exists a tradeoff between the quantity of memory resources invested, and the ease of language processing.
The shape of this tradeoff depends on the grammar of a language, and in particular the way that it structures information in time.
We characterize memory resources using the theory of lossy data compression \citep{cover2006elements,berger2003rate}.%, which has recently had success in predicting cognitive phenomena \citep{sims2018efficient,zenon2019information}.%, and we show how the tradeoff described by our general theory is manifested in different ways in existing mechanistic theories of language processing. 

Within our framework, we prove a theorem showing that lower memory requirements result when utterance elements that depend on each other statistically are placed close to each other. 
This theorem does not require any assumptions about the architecture or functioning of memory, except that it has a bounded capacity.
Using this concept, we introduce the \key{Efficient Tradeoff Hypothesis}: Order in natural language is structured so as to provide efficient memory--surprisal tradeoff curves.
We provide evidence for this hypothesis in three studies.
We demonstrate that word orders with short dependencies do indeed engender lower working memory resource requirements in toy languages studied in the previous literature, and we show that real word orders in corpora of 54 languages have lower memory requirements than would be expected under artificial baseline comparison grammars. 
Finally, we show that we can predict the order of morphemes within words in two languages using our principle of the minimization of memory usage.

Our work not only formalizes and tests an old idea in functional linguistics and psycholinguistics, it also opens up connections between those fields and the statistical analysis of natural language \citep{debowski-excess-2011,bentz2017entropy,lin-critical-2017}, and more broadly, between linguistics and fields that have studied information-processing costs and resource requirements in brains \citep[e.g.,][]{friston2010free} and general physical systems \citep[e.g.,][]{still2012thermodynamic}. 








\section{Background}\label{sec:background}


A wide range of work has argued that information in natural language utterances is ordered in ways that reduce memory effort, by placing elements close together when they depend on each other in some way. Here, we review these arguments from linguistic and cognitive perspectives.

\subsection{Dependency locality and memory constraints in psycholinguistics}

When producing and comprehending language in real time, a language user must keep track of what she has already produced or heard in some kind of incremental memory store, which is subject to resource constraints.
An early example of this idea is \citet{miller-finitary-1963}, who attributed the unacceptability of multiple center embeddings in English to limitations of human working memory.
Concurrent and subsequent work studied how different grammars induce different memory requirements in terms of the number of symbols that must be stored at each point to produce or parse a sentence \citep{yngve1960model,abney1991memory,gibson1991computational,resnik1992left}. 
In psycholinguistic studies, memory constraints typically manifest in the form of processing difficulty associated with long-term dependencies.
For example, at the level of word-by-word online language comprehension, there is observable processing difficulty at moments when it seems that information about a word must be retrieved from working memory. 
This difficulty increases when there is a great deal of time or intervening material between the point when a word is first encountered and the point when it must be retrieved from memory  \citep{gibson1998linguistic,gibson1999memory,gibson2000dependency,mcelree2000sentence,lewis-activation-based-2005,bartek-search-2011,nicenboim2015working,balling2017effects}. 
That is, language comprehension is harder for humans when words which depend on each other for their meaning are separated by many intervening words.
This idea is most prominently associated with the \key{Dependency Locality Theory} of human sentence processing \citep{gibson2000dependency}.

For example, \citet{grodner-consequences-2005} studied word-by-word reading times in a series of sentences such as~(\ref{ex:grodner}) below. 
\eenumsentence{\label{ex:grodner}
\item The \underline{administrator} who the nurse \underline{supervised}\dots\label{ex:grodner1}
\item The \underline{administrator} who the nurse from the clinic \underline{supervised}\dots
\item The \underline{administrator} who the nurse who was from the clinic \underline{supervised}\dots\label{ex:grodner3}
}
In these sentences, the distance between the noun \emph{administrator} and the verb \emph{supervised} is successively increased. \citet{grodner-consequences-2005} found that as this distance increases, there is a concomitant increase in reading time at the verb \emph{supervised} and following words. The hypothesized reason is the following: at the word \emph{supervised}, a comprehender who is trying to compute the meaning of the sentence must integrate a representation of the verb \emph{supervised} with a representation of the noun \emph{administrator}, which is a direct object of the verb. This integration requires retrieving the representation of \emph{administrator} from working memory. If this representation has been in working memory for a long time---for example as in Sentence~\ref{ex:grodner3} as opposed to \ref{ex:grodner1}---then the retrieval operation \jd{not remembering the details: it's specifically the retrieval operation that they say causes difficulty? not the long maintenance of "administrator" while other elements pile up in working memory?} causes difficulty that manifests as increased reading time. There exists a dependency between the words \emph{administrator} and \emph{supervised}, and more excess processing difficulty is incurred the more the two words are separated. 

%The existence of dependency locality effects in human language processing, and their connection with working memory, are well-established. In this work, we will be interested in predicting word and morpheme order using an information-theoretic model of language processing which accounts for locality effects such as these. \jd{why? why does it follow from the existence of dependency locality effects in processing that we should want to predict word and morpheme order, and that we should want to do so using an information-theoretic model? if this isn't the right place for this, i'd just get rid of this paragraph altogether and go straight into 2.2. in fact, reading the first sentence of 2.2., i think its better to go straight into it without this paragraph. perhaps just add the first sentence of this paragraph to the end of the previous paragraph (ideally with citations). but if the point is simply to say "Locality effects such as these fall out of the information-theoretic account we propose", just say that.}

\subsection{Locality and cross-linguistic universals of order}

Because of the documented difficulty associated with long-term dependencies among words, it has been hypothesized that working memory limitations create a pressure for dependency locality in word order. 
That is, words which depend on each other syntactically should be close to each other in linear order.
There is ample evidence from corpus statistics indicating that dependency locality is a real property of word order across many languages \citep{gildea-optimizing-2007,liu2008dependency,gildea-grammars-2010,futrell-large-scale-2015,liu-dependency-2017,temperley-minimizing-2018}. 
\citet{hawkins-performance-1994,hawkins-efficiency-2003} formulates dependency locality as the Principle of Domain Minimization, and has shown that this principle can explain cross-linguistic universals of word order that have been documented by linguistic typologists for decades \citep{greenberg-universals-1963}. 
An example is order alternation in postverbal constituents in English.
While NP objects ordinarily precede PPs (\ref{ex:heavy-np-1}, example from~\citet{staub2006heavy}), this order is less preferred when the NP is very long (\ref{ex:heavy-np-2}), in which case the inverse order becomes more felicitous (\ref{ex:heavy-np-3}).
The pattern in (\ref{ex:heavy-np-3}) is known as Heavy NP Shift \citep{ross1967constraints,arnold2000heaviness,stallings2011s}.
Compared to (\ref{ex:heavy-np-2}), it reduces the distance between the verb ``ate'' and the PP, while only modestly increasing the distance between the verb and object NP.

\eenumsentence{\label{ex:heavy-np}
\item Lucy ate [the broccoli] with a fork.\label{ex:heavy-np-1}
\item Lucy ate [the extremely delicious, bright green broccoli] with a fork .\label{ex:heavy-np-2}
\item Lucy ate with a fork [the extremely delicious, bright green broccoli].\label{ex:heavy-np-3}
}

%TODO dependency locality examples \mhahn{how about a heavy NP shift example?} \rljf{Yes, the ``threw out the trash'' example should owrk here} \mhahn{can use real Heavy NP shify example}

%While dependency locality has a strong ability to predict word order universals, there are certain classes of locality phenomena that are not captured by the theory. For example, ordering asymmetries between arguments and adjuncts remain unexplained \citep{}. TODO \mhahn{what is missing here?}

Locality principles have also appeared in a more general form in the functional linguistics literature, in the form of the idea that elements which are more `relevant' to each other will appear closer to each other in linear order in utterances \citep{behaghel1932deutsche,givon1985iconicity,givon1991markedness,bybee-morphology-1985,newmeyer1992iconicity}. Here, `elements' can refer to words or morphemes, and the definition of `relevance' varies. For example, \citet{givon1985iconicity}'s \key{Proximity Principle} states that elements are placed closer together in a sentence if they are closer conceptually.
Applying a similar principle, \citet{bybee-morphology-1985} studied the order or morphemes within words across languages, and argued that (for example) morphemes that indicate the valence of a verb (whether it takes zero, one, or two objects) are placed closer to the verb root than morphemes that indicate the plurality of the subject of the verb, because the valence morphemes are more `relevant' to the verb root. %In previous literature, the Proximity Principle has been motivated in terms of iconicity; here, we will explain it in terms of the same kinds of online memory limitations that have been used to motivate the principle of dependency locality in syntax.

While these theories are widespread in the linguistics literature, there exists to date no quantifiable definition of `relevance' or `being closer cenceptually'. One of our contributions is to derive such a notion of `relevance' from the minimization of memory usage during language processing. \jd{make sure this relevance point is taken back up in the general discussion}


\subsection{Cognitive Underpinnings}
\jd{change section header to sth like "Cognitive architecture" or "Architectural assumptions" or "Memory architecture"?}

The connection between memory resources and locality principles relies on the idea that limitations in working memory will give rise to difficulty when elements that depend on each other are separated at a large distance in time. In previous work, this idea has been motivated in terms of specific assumptions about the architecture of memory. For example, models of memory in sentence processing differ in whether they assume limitations in storage capacity \citep[e.g.][]{gibson1998linguistic} \jd{here, gibson is portrayed as a capacity theory; above it's portrayed as a retrieval theory} or the precision with which specific elements can be retrieved from memory \citep[e.g.][]{lewis-activation-based-2005}. Furthermore, in order to derive the connection between memory usage in such models and locality in word order, it has been necessary to stipulate that memory representations or activations decay over time in some way to explain why longer dependencies are harder to process.
The question remains of whether these assumptions about memory architecture are necessary, or whether word orders across languages are optimized for memory independently of the implementation and architecture of human language processing.

In this work, we adopt an information-theoretic perspective on working memory\jd{someone might say that the assumption that this is an account of working memory rather than different types of memory is already an architectural assumption} which abstracts away from the details of memory architecture.
%From an information-theoretic perspective, the different ways in which memory can be constrained---constraints on storage capacity, representation precision, etc.---are different instantiations of the same quantity: namely, the mutual information between past input and the representations that are used for processing new material \citep{still-information-2014}. \mhahn{do we need to say more here?}
Within our framework, we will establish the connection between memory resources and locality principles by providing general information-theoretic lower bounds on memory load.
%We will consider a general setting of a listener performing incremental prediction.
Our result immediately entails a link between locality and boundedness of memory, without any stipulations\jd{isn't "without any stipulations" too strong? the main stipulation is that the relevant memory representation is bits of information -- seems like a minor stipulation (and it should be noted somewhere that this subsumes different, more mechanistic assumptions about the nature of memory representations (right?), but a stipulation nonetheless, and one that's easy to acknowledge} about memory representations, and in particular without any assumption that memory representations or activations decay over time \citep[as was required in][]{gibson1998linguistic, lewis-activation-based-2005, futrell2020lossy}.
We will then show empirical evidence that the orders of words and morphemes in natural language are structured in a way that reduces our measure of memory load compared to the orders of counterfactual baseline languages.


The remainder of the paper is structured as follows. In Section 3, we introduce the memory-surprisal tradeoff, show that information locality is a property of languages with more efficient memory-surprisal tradeoffs, and introduce the main hypothesis that emerges from the account: the Efficient Tradeoff Curve Hypothesis. Sections 4, 5, and 6 are dedicated to testing the Efficient Tradeoff Curve Hypothesis. In Section 4, we qualitatively test the Hypothesis in a reanalysis of word orders emerging in a miniature artificial language study \CITE.
In Section 5, we quantitatively test the Hypothesis in a large-scale study of the word order of 54 languages. In Section 6, we test the Hypothesis on morpheme order in Japanese and Sesotho.
Finally, in Section 7 we discuss the implications and limitations of the reported results."


\jd{writing the above, i realized that it would be really useful if there was a way to introduce the ETCH in some informal way towards the very beginning of the paper, in the intro. i know this is probably difficult, but if there's an intuitive way to formulate the prediction (and simply defer the formal definition to section 3), i think that would make reading the background easier -- because we know what we're working towards -- and would also make writing the proposed forward-looking section a little easier to write}

%\mhahn{can also mention \cite{christiansen2016now}}

% Production argument
% Idea: A producer wants to approximate a language conditional on a production target G.
% Ie producer finds m_t to minimize loss: D[ w_t | w_{<t}, g   ||   w_t  | m_t  ] + a * H[m_t]
% That is, the memory has to contain both the previous words and the current production goal.
% The loss comes out to I[ w_t : w_{<t},g | m_t] + a * H[m_t]
%                       = I[w_t : w_{<t} | m_t] + I[w_t : g | w_{<t}, m_t] + a * H[m_t]
% Now let's decompose the memory m_t into two parts:
% the part about the previous words: call this r_t
% the part about the current goal: call this g_t. And let's assume H[g|g_t]=0, i.e. the memory stores
% all the information about the goal.
% Then H[m_t] = H[r_t] + H[g_t|r_t].
% Now we have loss:
%     I[w_t : w_{<t} | r_t, g] + I[w_t : g | w_{<t}, r_t, g] + a H[r_t, g_t]
%   = I[w_t : w_{<t} | r_t, g] + a H[r_t, g_t] 
% Now we have H[r_t, g_t] >= H[r_t] >= \sum t I_t.
% And then we have 





\section{Memory-Surprisal Tradeoff}\label{sec:ms-tradeoff}

In this section, we introduce the main concept and hypothesis of the paper. % from a psycholinguistic perspective. 
We provide a technical definition of the memory--surprisal tradeoff curve, and we prove a theorem showing that more efficient memory--surprisal tradeoffs are possible in languages exhibiting information locality, i.e., in languages where words that depend on each other are close to each other. This theorem establishes the formal link between memory efficiency in online processing and locality in word order.

%The memory--surprisal tradeoff applies in both language production and comprehension---more generally, it applies to any system that incrementally produces sequences or extracts information from them. Below, we will develop the theory first in terms of language comprehension; we return to the production perspective in the General Discussion. % (Section~\ref{sec:discussion}).



\subsection{An information-theoretic model of online language comprehension}
\label{sec:listener-tradeoff}

We begin developing our model by considering the process of language comprehension, where a listener is processing a stream of words uttered by an interlocutor.
Experimental research has established three properties of online language comprehension: (1) listeners maintain some information about the words received so far in incremental memory, (2) listeners form probabilistic expectations about the upcoming words \citep[e.g.][]{altmann1999incremental,staub2006syntactic,kuperberg2016we}, and (3) words are easy to process to the extent that they are predictable based on a listener's memory of words received so far \citep{hale2001probabilistic,levy2008expectation,futrell2020lossy}. 
See General Discussion % (Section~\ref{sec:discussion})
for discussion of how our model is related to theories that do not explicitly make these assumptions.

We formalize these three observations into postulates intended to provide a simplified picture of what is known about online language comprehension. Consider a listener comprehending a sequence of words $w_1, \dots, w_t, \dots, w_n$, at an arbitrary time $t$.
\begin{enumerate}
    \item Comprehension Postulate 1 (Incremental memory). At time $t$, the listener has an incremental \key{memory state} $m_t$ that contains her stored information about previous words. The memory state is characterized by a \key{memory encoding function} $M$ such that $m_t = M(w_{t-1}, m_{t-1})$.
    \item Comprehension Postulate 2 (Incremental prediction). The listener has a subjective probability distribution at time $t$ over the next word $w_t$ as a function of the memory state $m_t$. This probability distribution is denoted $P(w_t|m_t)$.
    \item Comprehension Postulate 3 (Linking hypothesis). Processing a word $w_t$ incurs difficulty proportional to the \key{surprisal} of $w_t$ given the memory state $m_t$:\footnote{\revision{In this paper, all logarithms are taken with base $2$. As choosing another base (e.g., $e$) would only result in multiplication with a proportionality constant, this assumption does not impact the generality of this linking hypothesis.}}
    \begin{equation}
    \label{eq:lossy-surp}
    \text{Difficulty} \propto -\log_2 P(w_t | m_t).
\end{equation}
\end{enumerate}
The claim that processing difficulty should be directly proportional to surprisal comes from \key{surprisal theory} \citep{hale2001probabilistic,levy2008expectation}, an established psycholinguistic theory that can capture reading time effects related to garden-path disambiguation, antilocality effects, and effects of syntactic construction frequency. Surprisal is a robust linear predictor of reading times in large-scale eye-tracking studies based on naturalistic text \citep{smith2013effect,goodkind-predictive-2018,frank2019interaction,aurnhammer2019evaluating,wilcox2020predictive}, and effects of surprisal have been observed for units as small as phonemes \citep{gwilliams2020neural}. There are several converging theoretical arguments for surprisal as a measure of processing cost \citep{levy2008expectation,smith2013effect}.
Surprisal theory is compatible with different views on the mechanisms underlying prediction, and can reflect different mechanisms such as preactivation and integration~\citep{kuperberg2016we}.
We do not assume that listeners explicitly compute a full-fledged distribution $P(w_t|m_t)$; we view $P(w_t|m_t)$ as a formalization of the probabilistic expectations that listeners form during comprehension.


Our expression~(\ref{eq:lossy-surp}) differs from the usual formulation of surprisal theory in that we consider predictability based on a (potentially lossy or noisy) memory representation $m_t$, rather than predictability based on the true complete context $w_1, \dots, w_{t-1}$. The generalization to lossy memory representations is necessary to capture empirically observed effects of memory limitations on language processing, such as dependency locality and structural forgetting \citep{futrell2020lossy}. 

In this work, we are interested in using theories of processing difficulty to derive predictions about languages as a whole, not about individual words or sentences. Therefore, we need a measure of the processing difficulty associated with a language as a whole. For this, we consider the \emph{average} surprisal per word in the language. We call this quantity the \key{average surprisal} of a language given a memory encoding function $M$, denoted $S_M$.

Crucially, the listener's ability to predict upcoming words accurately depends on how much she remembers about previous words. As the precision of her memory increases, the accuracy of her predictions also increases, and the average surprisal $S_M$ for each incoming word decreases. Taking an information-theoretic perspective, we can think about the amount of information (measured in bits) about previous words stored in the listener's memory state. This quantity of information is given by the \key{entropy} of the memory state, which we denote $H_M$. As the listener stores more and more bits of information about the previous words her memory state, she can achieve lower and lower surprisal values for the upcoming words. This tradeoff between memory and surprisal is the main object of study in this paper.

The \key{memory--surprisal tradeoff curve} answers the question: 
for a given amount of information about previous words $H_M$ stored in the listener's memory state, what is the lowest achievable average surprisal $S_M$? Two example tradeoff curves are shown in Figure~\ref{fig:examples}. In general, as the listener stores more information about previous words in her memory state, her lowest achievable average surprisal can only decrease. So the curve is always monotonically decreasing. However, the precise shape of the tradeoff curve depends on the structure of the language being predicted. For example, Figure~\ref{fig:examples} shows how two hypothetical languages might engender different tradeoff curves, with Language $A$ allowing more favorable tradeoffs than Language $B$. That is, for Language $A$, it is possible to achieve lower processing difficulty while investing less memory resources than in Language $B$.

\begin{figure}
\centering
\includegraphics[width=0.5\textwidth]{figures-gdrive/tradeoff-schematic.pdf}
	\caption{Example memory--surprisal tradeoff curves for two languages, $A$ and $B$. Achieving an average surprisal of 3.5 bits requires storing at least 1.0 bits in language $A$, while it requires storing 2.0 bits in language $B$. \revision{Language $A$ has a steeper memory--surprisal tradeoff than Language $B$}, and requires less memory resources to achieve the same level of processing difficulty.}
\label{fig:examples}
\end{figure}

\subsection{Main hypothesis}

Having conceptually introduced the memory--surprisal tradeoff, we can state the main hypothesis of this work, the Efficient Tradeoff Hypothesis.

\begin{adjustwidth}{6em}{6em}
\textbf{Efficient Tradeoff Hypothesis:}\\
The order of elements in natural language is characterized by a distinctively steeper memory--surprisal tradeoff curve, compared to other possible orders.
\end{adjustwidth}
A steep tradeoff curve corresponds to memory efficiency, in the sense that it is possible to achieve a low level of processing difficulty (average surprisal $S_M$) while storing a relatively small amount of information about previous words (entropy of memory $H_M$).
We hypothesize that this property is reflected in grammatical structure and usage preferences across languages.


\subsection{Formal definition of the memory--surprisal tradeoff}
\label{sec:formal-tradeoff}

Here we provide the technical definition of the memory--surprisal tradeoff curve.
Let $W$ be a stochastic process generating a stream of symbols extending indefinitely into the past and future: $\dots, w_{-2}, w_{-1}, w_0, w_{1}, w_{2}, \dots$. %indexed as $w_1, \dots, w_t, \dots$.
These symbols can represent words, morphemes, or other units for decomposing sentences into a sequence of symbols. 
We model this process as stationary \citep{doob1953stochastic}, that is, the joint probability distributions of symbols at different time points depend only on their relative positions in time, not their absolute positions (see SI Section 1.1.1 for more on this modeling assumption).

Let $M$ be a memory encoding function.
We consider memory and surprisal costs at an arbitrary time point $t$.
Recall that the surprisal for a specific word $w_t$ after a past word sequence $\dots, w_{t-2}, w_{t-1}$ encoded into a memory state $m_t$ is:
\begin{equation*}
    -\log_2 P(w_t | m_t).
\end{equation*}
The \key{average surprisal} of the process $W$ under the memory encoding function $M$ is obtained by averaging over all possible past sequences $\dots, w_{t-2}, w_{t-1}$ with associated memory states $m_t$, and possible next words $w_t$:
\begin{equation*}
   S_M \equiv -\sum_{w_t,m_t}  P(m_t) P(w_t|m_t) \log_2 P(w_t | m_t).
\end{equation*}
where $w_t$ ranges over possible symbols, and $m_t$ ranges over possible outputs of the memory encoding function $M$.
This quantity is known as the conditional entropy of $w_t$ given $m_t$ \citep[][p. 17]{cover2006elements}:
\begin{equation*}
	S_M = \operatorname{H}[w_t | m_t].
%    S_M \equiv \lim_{T\rightarrow\infty} \frac{1}{T} \sum_{t=1}^T \operatorname{H}[w_t | m_t],
\end{equation*}
%where the notation $\operatorname{H}[\cdot | \cdot]$ indicates conditional entropy \citep[][p. 17]{cover2006elements}:
%\begin{equation}
%    \operatorname{H}[w_t|m_t] \equiv -\sum_{w_t,m_t} P(m_t) P(w_t|m_t) \log P(w_t|m_t).
%\end{equation}
Because the process $W$ is stationary, the average surprisal $S_M$ is independent of the choice of $t$ (see SI Section 1.1.2).
The lowest possible average surprisal for $W$ is attained when $m_t$ perfectly encodes all previous observed words.
This quantity is called the \key{entropy rate} of $W$ \citep[][pp. 74--75]{cover2006elements}:
\begin{equation*}
    %\label{eq:entropy-rate}
	S_\infty \equiv \operatorname{H}[w_t | \dots, w_{t-2}, w_{t-1}],
%    S_\infty \equiv \lim_{t \rightarrow \infty} \frac{1}{T} \sum_{t=1}^T H[w_t | w_1, \dots, w_{t-1}].
\end{equation*}
which again is independent of $t$ because $W$ is stationary.
We use the notation $S_\infty$ to suggest this idea of unlimited resources.
 The entropy rate of a stochastic process is the irreducible unpredictability of the process: the extent to which a stream of symbols remains unpredictable even for a predictor with unlimited resources. 

 Because the memory state $m_t$ is a function of the previous words $ \dots, w_{t-2}, w_{t-1}$, we can prove by the Data Processing Inequality \citep[][pp. 34--35]{cover2006elements} that the entropy rate must be less than or equal to the average surprisal for any memory encoding function $M$:
\begin{equation*}
    %\label{eq:entropy-rate-dpi}
    S_\infty \le S_M.
\end{equation*}


If the memory state $m_t$ stores all information about the previous words $ \dots, w_{t-2}, w_{t-1}$, then we have $S_M = S_\infty$.
%Based on Eq.~\ref{eq:entropy-rate-dpi}, we can write the average surprisal $S_M$ as a sum of two non-negative terms,
%\begin{equation}
%    S_M = S_\infty + d_M,
%\end{equation}
%where $d_M$ is \key{memory distortion}: the extra surprisal incurred in addition to the unavoidable surprisal $S_\infty$, owing to the lossiness of the memory encoding function $M$. 
%The memory distortion $d_M$ is formally a Kullback-Leibler (KL) divergence:
%\begin{equation}
%    \label{eq:memory-distortion}
%    d_M = \lim_{t \rightarrow \infty} D_{\text{KL}} [ p(w_t | w_1, \dots, w_{t-1}) || p(w_t | m_t)].
%\end{equation}
%Finding a memory encoding function $M$ to minimize $S_M$ is equivalent to minimizing the memory distortion $d_M$.
%\mhahn{do we need to introduce the distortion $d_M$?}

Having defined average surprisal, we now turn to the question of how to define memory capacity. The average amount of information stored in the memory states $m_t$ is the average number of bits required to encode $m_t$. 
This is given by the entropy of the stationary distribution over memory states, $H_M$: %, again averaged over all time points:
\begin{align}
    %\label{eq:memory-entropy}
    \nonumber
        H_M &\equiv \operatorname{H}[m_t] 
    %H_M &\equiv \lim_{T\rightarrow\infty} \frac{1}{T} \sum_{t=1}^T H[m_t] \\
\end{align}
where
\begin{equation*}
    \operatorname{H}[m_t] = - \sum_m P(m_t = m) \log_2 P(m_t=m)
\end{equation*}
where $m$ runs over all possible states of the memory encoding $m_t$.
Again, because $W$ is stationary, this quantity does not depend on the choice of $t$ (see SI Section 1.1.2).


We will be imposing bounds on $H_M$ and studying the resulting values of $S_M$. 

\begin{definition}
The \key{memory--surprisal tradeoff curve} for a process $W$ is the lowest achievable average surprisal $S_M$ for each value of $H_M$. Let $R$ denote an upper bound on the memory entropy $H_M$; then the memory--surprisal tradeoff curve as a function of $R$ is given by
\begin{equation}
    \label{eq:ms-formal}
    D(R) \equiv \min_{M : H_M \le R} S_M,
\end{equation}
where the minimization is over all memory encoding functions $M$ whose entropy $H_M$ is less than or equal to $R$.
\end{definition}

The memory state $m_t$ is generally a lossy representation of the true context of words $w_1, \dots, w_{t-1}$, meaning that $m_t$ does not contain all the possible information about $w_1, \dots, w_{t-1}$. The mathematical theory of lossy representations is \key{rate--distortion theory} \citep[for an overview and key results, see][pp. 301--347]{cover2006elements}; this theory has seen recent successful application in cognitive science and linguistics as a model of rational action under resource constraints \citep{brady2009compression,sims2012ideal,sims2018efficient,zaslavsky2018efficient,schach2018quantifying,zenon2019information,gershman2020origin}. 
Rate--distortion theory studies curves of the form of Eq.~\ref{eq:ms-formal}, which quantify tradeoffs between negative utility (`distortion') and information (`rate'). %Our memory--surprisal tradeoff curve is a distortion--rate curve, with $H_M$ as the rate and $S_M$ as the distortion. It differs from the typical curve studied in rate--distortion theory in that we define rate using entropy, rather than mutual information \citep[see][for a discussion of some of the consequences of this formulation]{strouse-deterministic-2017}. 
%Our theoretical results still hold if we were to define rate using mutual information (see SI Section \REF).


\begin{figure}
	(a)
\includegraphics[width=0.4\textwidth]{figures-gdrive/mi-distance.pdf}
	(b)
\includegraphics[width=0.25\textwidth]{figures-gdrive/theorem.pdf}
	\caption{
		(a) Conditional mutual information $I_t$ captures how much predictive information about the next word is provided, on average, by the word $t$ steps in the past.
		(b) Here we illustrate our theoretical result. We plot $I_t$ (top) and $tI_t$ (bottom) as functions of $t$. For any choice of $T$, a listener using $B$ bits memory (bottom) to represent prior observations will incur at least $A$ bits of extra surprisal beyond the entropy rate (top). 
}\label{fig:theorem}
\end{figure}

\subsection{Information locality}
\label{sec:infoloc}\label{sec:tradeoff}

The shape of the memory--surprisal tradeoff is determined in part by the grammatical structure of a language.
Some \revision{hypothetical} languages enable more efficient tradeoffs than others by allowing a listener to store fewer bits in memory to achieve the same level of average surprisal.

Here, we will demonstrate that the memory--surprisal tradeoff is optimized by languages with word orders exhibiting a property called \key{information locality}. Information locality means that words that depend on each other statistically are located close to each other in time. We will argue that information locality generalizes the well-known word order principle of dependency locality.

We will make our argument by defining a lower bound on the memory--surprisal tradeoff curve (Eq.~\ref{eq:ms-formal}). This lower bound represents an unavoidable cost associated with a certain level of memory usage $H_M$; the true average surprisal $S_M$ might be higher than this bound. 

Our argument will make use of a quantity called \key{mutual information}. Mutual information is the most general measure of statistical association between two random variables. The mutual information between two random variables $X$ and $Y$, conditional on a third random variable $Z$, is defined as:
\begin{align}
\label{eq:mi}
    \operatorname{I}[X:Y|Z] &\equiv \sum_{x,y,z} P(x,y,z) \log_2 \frac{P(x,y|z)}{P(x|z)P(y|z)}. % \text{ bits} \\
    %\nonumber
    %&= \operatorname{H}[X|Z] - \operatorname{H}[X|Y,Z] \\
    %\nonumber
    %&= \operatorname{H}[Y|Z] - \operatorname{H}[Y|X,Z].
\end{align}
Mutual information is always non-negative. It is zero when $X$ and $Y$ are conditionally independent given $Z$, and positive whenever $X$ gives any information that makes the value of $Y$ more predictable, or vice versa. 

We will study the mutual information structure of natural language sentences, and in particular the mutual information between words at certain distances in linear order. We define the notation $I_t$ to mean the mutual information between words at distance $t$ from each other, conditional on the intervening words:
\begin{equation*}
    I_t \equiv \operatorname{I}[w_t : w_0 | w_1, \dots, w_{t-1}].
\end{equation*}
This quantity, visualized in Figure~\ref{fig:theorem}(a), measures how much predictive information is provided about the current word by the word $t$ steps in the past.
It is a statistical property of the language, and can be estimated from large-scale text data.

Equipped with this notion of mutual information at a distance, we can now state our theorem:
\begin{thm}\label{prop:suboptimal}(Information locality bound)
For any positive integer $T$, let $M$ be a memory encoding function such that
\begin{equation}
\label{eq:memory-bound}
H_M \le \sum_{t=1}^T t I_t.    
\end{equation}
Then we have a lower bound on the average surprisal under the memory encoding function $M$:
\begin{equation}
\label{eq:surprisal-bound}
S_M \ge S_\infty + \sum_{t=T+1}^\infty I_t.
\end{equation}
\end{thm}
A formal proof based on the Comprehension Postulates 1--3 is given in SI Section 1.2.
\revision{An intuitive argument, forming the basis of the proof, is the following.
Suppose that a comprehender predicting the $t$'th word $w_t$ uses an average of $I_t$ bits of information coming from the previous word $w_0$. 
Then these bits must have been carried over $t$ timesteps and thus have occupied memory for $t$ timesteps.
Since this happens for every word in a sequence, there are, at any given point in time, $t$ such packets of information, each with an average size of $I_t$ bits, that have to be maintained, summing up to $t I_t$.
In the specific setting where $M$ encodes information from a contiguous span of the past $T$ words, the total amount of encoded information thus sums up to $\sum_{t=1}^T t I_t$, while information from longer contexts is lost, increasing surprisal by $\sum_{t=T+1}^\infty I_t$.
While this informal argument specifically considers a memory encoding function that utilizes information from a contiguous span of the past $T$ words, the formal proof extends this to all memory encoding functions $M$ satisfying the Comprehension Postulates.}

\paragraph{Interpretation} The theorem means that a predictor with limited memory capacity will always be affected by surprisal cost arising from long-term statistical dependencies of length greater than $T$, for some finite $T$. This is why we call the result `information locality': processes are easier to predict when most statistical dependencies are short-term (shorter than some $T$). Below we explain in more detail how this interpretation matches the mathematics of the theorem.

The quantities in the theorem are illustrated visually in Figure~\ref{fig:theorem}. Eq.~\ref{eq:memory-bound} describes a memory encoding function which has enough capacity to remember the relevant information from at most $T$ words in the immediate past. The minimal amount of memory capacity which would be required to retain this information is the sum $\sum_{t=1}^T t I_t$, reflecting the cost of holding $I_t$ bits in memory for $t$ timesteps up to $t=T$. 

The information locality bound theorem says that the surprisal cost for this memory encoding function is at least $S_\infty + \sum_{t=T+1}^\infty I_t$ (Eq.~\ref{eq:surprisal-bound}). The first term $S_\infty$ is the entropy rate of the process, representing the bits of information in the process which could not have been predicted given any amount of memory. The second term $\sum_{t=T+1}^\infty I_t$ is the sum of all the relevant information contained in words \emph{more} than $T$ timesteps in the past (see Figure~\ref{fig:theorem}(b)). These correspond to bits of information in the process which \emph{could have} been predicted given infinite memory resources, but which were not, due to the limit on memory usage.

The theorem gives a lower bound on the memory--surprisal tradeoff curve, meaning that there is no memory encoding function $M$ with capacity $H_M$ which achieves lower average surprisal than Eq.~\ref{eq:surprisal-bound}. In terms of psycholinguistics, if memory usage is bounded by Eq.~\ref{eq:memory-bound}, then processing cost of at least Eq.~\ref{eq:surprisal-bound} is inevitable.
Importantly, the bound holds for \emph{any} memory encoding function $M$, including functions that do not specifically keep track of a window of the past $T$ words.
The information locality bound theorem demonstrates in a highly general way that language comprehension requires less memory resources when statistical dependencies are mostly short-term. 

\begin{figure*}
\includegraphics[width=0.45\textwidth]{figures/decay.pdf}
\includegraphics[width=0.45\textwidth]{figures/memory.pdf}
%
	\caption{Left: $I_t$ as a function of $t$, for two different hypothetical languages. $I_t$ decays faster for the MoreEfficient language: Predictive information about the present observation is concentrated more strongly in the recent past. Right: $t \cdot I_t$ as a function of $t$ for the same languages. }\label{fig:basic}
\end{figure*}

\begin{figure}
\includegraphics[width=0.45\textwidth]{figures/listener-tradeoff.pdf}
	\caption{Listener's memory--surprisal tradeoff for the two hypothetical languages in Figure~\ref{fig:basic}. Recall that the MoreEfficient language has a faster decay of conditional mutual information $I_t$. Correspondingly, this figure shows that a listener can achieve lower average surprisal at the same level of memory load.}\label{fig:listener-tradeoff}
\end{figure}

%Eq.~\ref{eq:memory-bound} means that maintaining long-term dependencies requires higher memory usage. Carrying the same amount of information over longer distances requires more memory; this fact is reflected in the factor $t$ inside each term of the sum. 
%The result is that modeling long-term statistical dependencies is more costly in terms of memory usage than modeling shorter ones; this cost can manifest in memory storage or in surprisal.


Because processing long-term dependencies requires higher memory usage, the theorem also implies that a language can be easier to process when most of the predictive information about a word is concentrated close to that word in time---that is, when $I_t$ falls off rapidly as $t \rightarrow \infty$. When memory capacity is limited, then there must be some timescale $T$ such that a listener appears to be affected by excess surprisal arising from statistical dependencies of length greater than $T$. A language avoids such cost to the extent that it avoids dependencies with a time-span larger than $T$.

We illustrate the theorem in Figure~\ref{fig:basic}.
We consider two hypothetical languages, LessEfficient and MoreEfficient, where $I_t := 5t^{-1.5}$ for LessEfficient and $I_t := 3.5 t^{-2.5}$ for MoreEfficient.\footnote{\revision{Although these are purely mathematical examples, the $I_t$ curve for natural languages does seem empirically to fall off as a power law as in these examples \citep{debowski2016relaxed}.}}
The curves of $I_t$, as a function of the distance $t$, are shown in Figure~\ref{fig:basic} (left).
In both cases, $I_t$ converges to zero as $t$ grows to infinity. 
However, $I_t$ decays more quickly for language MoreEfficient.
This means that predictive information about an observation is concentrated more strongly in the recent past.
In Figure~\ref{fig:basic} (right), we show $t\cdot I_t$ as a function of $t$.
Note that the area under the curve is equal to (\ref{eq:memory-bound}).
This area is smaller for the MoreEfficient language, as $I_t$ decays more quickly there.  
In Figure~\ref{fig:listener-tradeoff}, we show the resulting bounds on memory--surprisal tradeoffs of the two languages.  
As $I_t$ decays faster for language MoreEfficient, it has a more efficient memory--surprisal tradeoff, allowing a listener to achieve strictly lower surprisal across a range of memory values.

\subsection{Other kinds of memory bottlenecks}

\revision{We derived the memory--surprisal tradeoff and the Information Locality Lower Bound by imposing a capacity limit on memory using the entropy $H_M$. The entropy $H_M$ represents the average amount of information that can be stored in memory at any time. However, in some psycholinguistic theories, memory-related difficulty arises not because of a bound on memory capacity, but rather because of difficulties involved in retrieving information from memory \citep{mcelree2000sentence,lewis-activation-based-2005,nicenboim2018models,vasishth2019computational}. }

\revision{It turns out that it is possible to derive results closely analogous to ours by imposing a capacity limit on the retrieval of information from memory, rather than the storage of information. Essentially, the constraint on the memory state in our Theorem~\ref{prop:suboptimal} can be re-interpreted as a constraint on the capacity of a communication channel linking short-term memory to working memory. This result constrains average surprisal for memory models based on cue-based retrieval such as the ACT-R model of \citet{lewis-activation-based-2005}. In fact, the theorem based on retrieval capacity gives a tighter bound than the theorem based on storage capacity. For the full model and derivation, see SI Section 1.3. }

\revision{We believe that concepts analogous to the memory--surprisal tradeoff and the Information Locality Lower Bound are likely to be valid across a broad range of models of incremental processing and memory.}


\section{Study 1: Memory and Dependency Length}\label{sec:toy-study}

So far, we have proven that there must exist a trade-off between memory and surprisal, and that this trade-off is optimized when languages have relatively short-term dependencies. Here, we illustrate the linguistic predictions of our theoretical results, and confirm that favored word orders with short syntactic dependencies do indeed optimize the memory--surprisal tradeoff in a toy language.

We will illustrate the operation of our theoretical framework by reanalyzing the data from \cite{fedzechkina-human-2017}.
This is a miniature artificial language study that showed a bias for dependency locality in production in artificial language learning. 
We will show that the languages which were favored in the artificial language learning experiment are those which optimize the memory--surprisal trade-off.

\subsection{Background: \citet{fedzechkina-human-2017}}

\citet{fedzechkina-human-2017} conducted a miniature artificial language learning experiment in which participants were exposed to videos describing simple events, paired with sentences in an articial language of the form Subject--Object--Verb or Object--Subject--Verb, in equal proportion, with free variation between these two word orders. The subject and the object were either simple nouns, or complex noun phrases with modifiers. Participants were trained to produce sentences in response to videos.

Crucially, \citet{fedzechkina-human-2017} set up the experiment such that in all training trials, other the subject and the object were both simple, or they were both complex. Then, after participants were sufficiently skilled in the use of the artificial language, they were asked to produce sentences describing videos with mixed complexity of noun phrases. The possible word orders that could be produced in this mixed-complexity setting are shown in Figure~\ref{tab:artificial}; the orders marked $A$ would create short dependencies, and the orders marked $B$ would create long dependencies. 
\begin{figure}
	\textbf{A Orders: Short Dependencies}

	OSV: [[Adjective Noun Postposition] Noun-\textsc{Case}] Noun Verb

	SOV: [[Adjective Noun Postposition] Noun] Noun-\textsc{Case} Verb

%
%	OSV:
%		\begin{tabular}{ccc}
%%			Object & Subject & Verb \\
%			\fbox{\begin{tabular}{llllll}
%				\fbox{\begin{tabular}{llll} Adjective &Noun &Postposition\end{tabular}} & Noun-\textsc{\textsc{Case}}
%					\end{tabular}} & \fbox{\begin{tabular}{l}Noun\end{tabular}} & \fbox{\begin{tabular}{l}Verb\end{tabular}}  \\
%		\end{tabular}
%\\
%
%SOV:
%		\begin{tabular}{ccc}
%			%			Subject & Object & Verb \\
%			\fbox{\begin{tabular}{llllll}
%				\fbox{\begin{tabular}{llll} Adjective &Noun &Postposition\end{tabular}} & Noun
%					\end{tabular}} & \fbox{\begin{tabular}{l}Noun-\textsc{\textsc{Case}}\end{tabular}} & Verb \\
%		\end{tabular}
%\\
%\\

	\textbf{B Orders: Long Dependencies}

	SOV: Noun [[Adjective Noun Postposition] Noun-\textsc{Case}] Verb

	OSV: Noun-\textsc{Case} [[Adjective Noun Postposition] Noun] Verb
%
%		\begin{tabular}{ccc}
%%			Subject & Object & Verb \\
%			 \fbox{\begin{tabular}{l}Noun\end{tabular}} &  \fbox{\begin{tabular}{llllll}
%				\fbox{\begin{tabular}{llll} Adjective &Noun &Postposition\end{tabular}} & Noun-Case
%		\end{tabular}}  &  \fbox{\begin{tabular}{l}Verb\end{tabular}}  \\
%		\end{tabular}
%\\
%		\begin{tabular}{ccc}
%%			Object & Subject & Verb \\
%			\fbox{\begin{tabular}{l}Noun-Case\end{tabular}} & \fbox{\begin{tabular}{llllll}
%				\fbox{\begin{tabular}{llll} Adjective &Noun &Postposition\end{tabular}} & Noun
%		\end{tabular}} & Verb \\
%		\end{tabular}
%
			\caption{Production targets in the miniature artificial language from \cite{fedzechkina-human-2017}. The language has head-final order, with free variation between SO and OS orders. When one of the arguments is much longer than the other, placing the longer one first ($A$ orders) shortens syntactic dependencies, compared to $B$ orders.}\label{tab:artificial}

\end{figure}

\citet{fedzechkina-human-2017} hypothesized, and indeed found, that participants favored the $A$ orders over the $B$ orders, despite the fact that there was no pattern in the participants' training input which would have favored $A$ over $B$. That is, when exposed to input which was ambiguous with respect to language $A$ or $B$, participants favored language $A$. \citet{fedzechkina-human-2017} explain the result in terms of dependency locality, because the $A$ orders create short dependencies between the verb and its arguments, and the $B$ orders create long dependencies.

\subsection{Calculating the memory--surprisal trade-off for the artificial languages}

We hypothesize, in line with our Main Hypothesis, that the favored language $A$ has a steeper memory--surprisal trade-off curve than the disfavored language $B$. Because of the controlled nature of this artificial language, we are able to test this hypothesis by exactly computing the bound on memory as given in Theorem~\ref{prop:suboptimal}. In fact, for this toy process, we can prove that the bound provided by the theorem is achievable, meaning that our computations reflect the true memory--surprisal trade-off curve, and not only a lower bound on it.

We only consider the head-final version of \citet{fedzechkina-human-2017}'s artificial language. This is because our bound on the memory--surprisal trade-off curve is invariant under reversal of a language. That is, if we take a language and reverse the order of all the words in all its sentences, we would measure the same lower bound on the memory--surprisal trade-off curve (for a proof, see SI Section \REF). Therefore, strictly head-final and strictly head-initial languages are equivalent under our bound.\footnote{Note that this invariance to reversal applies only to our \emph{lower bound} on the memory-surprisal trade-off curve; the true curve is not generally invariant to word order reversal \citep{crutchfield-times-2009}.} 

%The language has consistent head-final order, and uses case marking on objects.
%The relevant production targets are transitive sentences where one of the two arguments is much longer than the other, due to the presence of a PP modifier, as shown in Table~\ref{tab:artificial}.
%The language has variable order of subjects and objects; for the production targets, the B versions produce much longer dependencies than the A versions.
%Dependency Length Minimization thus predicts that speakers are more likely to use the A versions.
%\cite{fedzechkina-human-2017} confirmed this experimentally.

We constructed a stochastic process representing the language consisting of sentences with the $A$ orders from Figure~\ref{tab:artificial}, and one language consisting of the $B$ orders. Following the experimental setup of \cite{fedzechkina-human-2017}, we assigned equal probability to the two possible configurations per language, and used a separate set of nouns (inanimate nouns) for the embedded noun in the long phrase.

% TODO: Does the process only include those orders, or does it also include the SOV and OSV orders with equal-complexity O and S?

We interpreted each of the two languages as a stationary processes, extending infinitely in both directions, by concatenating independent samples drawn from the language.
We computed (\ref{eq:memory-bound}) from a chain of 1000 independently sampled sentences, for each of the two versions of the toy language.
% TODO: More on how this was done.			
% TODO: Convert nats to bits

\subsection{Results}	

\begin{figure*}
\centering
\includegraphics[width=0.45\textwidth]{figures/toy-mis.pdf}
\includegraphics[width=0.45\textwidth]{figures/toy-t-mis.pdf}
%
	\caption{Left: Decay of conditional mutual information $I_t$, as a function of the distance $t$, for the two versions in the artificial language. The areas under the two curves are identical, corresponding to the fact that both orders are equally predictable. However, mutual information decays faster in language $A$.\ \ \ Right: The minimal memory requirement $t I_t$ to store $I_t$ bits of information for timespan $t$, as a function of $t$. The area under the $B$ curve is larger, corresponding to larger memory demand for this order.}\label{fig:toy-mis}
\end{figure*}

Figure~\ref{fig:toy-mis} (left) shows the curve of the conditional mutual information $I_t$ as a function of the distance $t$, for the two languages $A$ and $B$. 
The curves differ at $t=2$ and $t=5$: 
About 0.073 units of predictive information that are at distance $t=2$ in the $A$ orders are moved to $t=5$ in the $B$ orders.
The source of the difference lies in predicting the presence and absence of a case marker on the second argument---i.e., whether to anticipate a subject or object.
In the $A$ orders, considering the last two words is sufficient to make this decision.
In the $B$ orders, it is necessary to consider the word before the long second constituent, which is five words in the past.

The total amounts of predictive information---corresponding to the area under the $I_t$ curve----are the same for both languages, indicating that both languages are equally predictable.
			
However, we will see that the memory demands are different.
Figure~\ref{fig:toy-mis} (right) shows the minimal memory requirements for remembering predictive information at a distance $t$ ($t\cdot I_t$) as a function of $t$.
As $I_t$ decays faster in $A$ orders, the total area under the curve now differs between $A$ and $B$, and is larger in $B$.
			Therefore, achieving the same predictive accuracy in language $B$ requires more memory resources than in language $A$.
			%This area corresponds to the lower bound in (\ref{eq:memory-bound}), and is 2.21 nats in A orders, and 2.43 nats in B orders.

%While (\ref{eq:memory-bound}) is a general lower bound, it can be proven that this bound is actually tight in the case of this specific example.\footnote{This can be shown by computing the causal states and then showing that the crypticity is zero, both of which is tractable in the case of this small-scale artificial language.}
%That is, a speaker who optimally allocates memory resources will spend 2.21 nats in A orders, and 2.43 nats in B orders.

\begin{figure}
\centering
\includegraphics[width=0.45\textwidth]{figures/toy-mem-surp.pdf}
	\caption{Tradeoff between listener memory and surprisal, for the two versions of the artificial language from \cite{fedzechkina-human-2017}. Language $A$ requires less memory at the same level of surprisal. }
	\label{fig:toy-listener-tradeoff}
\end{figure}

In Figure~\ref{fig:toy-listener-tradeoff}, we show the resulting memory--surprisal trade-off curve for the two versions of the artificial language from \cite{fedzechkina-human-2017}.
The curve shows that, at any desired level of surprisal, language $A$ requires at most as much memory as language $B$.
For reaching optimal surprisal, the empirically-favored language $A$ requires strictly less memory.


%It is important to stress that, even though we computed this value by considering the number of words impacting predictions at a given point in time, this bound holds independently of the actual implementation and architecture of memory and predictions.





%Both languages have the same overall entropy rate, but they differ in the distribution of predictive information.
%
%plot of $I_t$
%
%The areas under the curves are identical.
%
%good version
%
%%CONTEXT LENGTH 0   2.06856758681  9997.93143241   0.0
%%CONTEXT LENGTH 1   0.29195106479  1.77661652202   1.77661652202
%%CONTEXT LENGTH 2   0.0729865489103  0.21896451588   2.21454555377
%%CONTEXT LENGTH 3   0.0729865489103  0.0   2.21454555377
%%CONTEXT LENGTH 4   0.0729865489103  0.0   2.21454555377
%%CONTEXT LENGTH 5   0.0729865489103  0.0   2.21454555377
%%CONTEXT LENGTH 6   0.0729865489103  0.0   2.21454555377
%
%bad version
%
%%CONTEXT LENGTH 0   2.06937301571  9997.93062698   0.0
%%CONTEXT LENGTH 1   0.291673373027  1.77769964269   1.77769964269
%%CONTEXT LENGTH 2   0.145830550934  0.145842822093   2.06938528687
%%CONTEXT LENGTH 3   0.145830550934  0.0   2.06938528687
%%CONTEXT LENGTH 4   0.145830550934  0.0   2.06938528687
%%CONTEXT LENGTH 5   0.0729152754672  0.0729152754672   2.43396166421
%%CONTEXT LENGTH 6   0.0729152754672  0.0   2.43396166421
%%CONTEXT LENGTH 7   0.0729152754672  0.0   2.43396166421
%%
%
%
%
%%
%%grammar:
%%
%%S $\rightarrow$ Obj Subj V (1/2) | Subj Obj V (1/2)
%%
%%Obj $\rightarrow$ NP di
%%
%%Subj $\rightarrow$ NP
%%
%%NP $\rightarrow$ N (3/4) | PP NP (1/8) | Adj NP (1/8)
%%
%%PP $\rightarrow$ NP P
%%
%
%




\subsection{Discussion}

In a reinterpretation of previous experimental findings, we showed that the languages which are favored in artificial language learning experiments are those which optimize the memory--surprisal trade-off. 
This is evidence that learners and/or speakers have a bias toward word orders that optimize the trade-off. 
Furthermore, this result solidifies the link between the memory--surprisal trade-off and more traditional notions from linguistics, such as dependency locality. 
We found that the word orders which are optimal from the perspective of dependency locality are also those orders which are optimal from the perspective of the memory--surprisal trade-off.




\section{Study 2: Large-Scale Evidence that Word Orders Optimize Memory-Surprisal Tradeoff}
\label{sec:main-experiment}

To test whether word orders as found in natural language reflect optimization for the memory--surprisal trade-off more generally, we compare the memory--surprisal trade-offs of 54 actual languages to those of counterfactual baseline languages. These baseline languages differ from the actual languages only in their word order rules. This method of comparison against counterfactual baseline languages was introduced by \citet{gildea-optimizing-2007,gildea-grammars-2010} and has since been fruitfully applied to study optimization-based models of word order universals  \citep{futrell-large-scale-2015,gildea-human-2015,hahn2020universals}.

Here, we describe how we measure the memory--surprisal trade-off in corpora, and how we generate counterfactual baseline languages. In Section~\ref{sec:main-experiment-results}, we compare the trade-off in real corpora against the trade-off in the counterfactual baselines. For the majority of languages, we find that the real languages have more favorable memory--surprisal trade-offs than the baselines, in line with the Efficient Tradeoff Hypothesis.

\subsection{Measuring the memory--surprisal trade-off in corpora}

The key to evaluating the memory-surprisal tradeoff from corpus data is the set of quantities $I_t$, the  mutual information between words at distance $t$ conditional on the intervening words (defined in Section~\ref{sec:infoloc}). 
These quantities can be plugged in to Theorem~\ref{prop:suboptimal} to give a lower bound on the memory--surprisal trade-off.

The quantities $I_t$ can be estimated as the difference between the average surprisal of Markov models that have access to the last $t-1$ or $t$ words.
That is, if we have a $t$th-order Markov model with average surprisal
\begin{equation*}
    S_t = H[w_t | w_0, \dots, w_{t-1}]
\end{equation*}
and a $(t-1)$th-order Markov model with average surprisal
\begin{equation*}
    S_{t-1} = H[w_t | w_1, \dots, w_{t-1}],
\end{equation*}
then we can calculate $I_t$ straightforwardly in the following way:
\begin{align}
    \nonumber
    I_t &= I[w_t : w_0 | w_1, \dots, w_{t-1}] \\
    \nonumber
    &= S_{t-1} - S_t.
\end{align}
Therefore, to evaluate $I_t$, all we need is a way of fitting Markov models of order $t$ and $t-1$ and computing their average surprisals.

To fit Markov models to the data, we use neural language models. In particular, we use Recurrent Neural Networks with Long Short-Term Memory architectures \citep{hochreiter-long-1997}. 
Neural network models are the basis of the state-of-the art in statistical modeling of language. Surprisal estimates derived from such models have been shown to best predict reading times, compared to other models, e.g., n-gram models~\citep{frank-insensitivity-2011,goodkind-predictive-2018}.
See Supplementary Materials Section 3.2 for details on how these models were fit to data, and see Supplementary Materials Sections 3.4 and 3.5 for control studies using other methods of estimating $I_t$ (based on $n$-gram models and PCFG chart parsers). These control studies yield the same qualitative results as the neural network-based studies presented here.

In order to evaluate the average surprisal values $S_t$, we computed the empirical word-by-word surprisal values under the $t$th-order Markov model for held-out data, different from the data that was used to train the model. By evaluating on held-out data, we avoid underestimating the value of $S_t$ due to overfitting.
We chose held-out data based on existing splits of corpora (see Section~\ref{sec:exp2-data}).
%\jd{say a tiny bit more about the model fitting procedure, eg how much data was held out}




\subsection{Data}\label{sec:exp2-data}
We draw on syntactically annotated corpora, compiled by the Universal Dependencies project for several dozen languages~\citep{nivre-universal-2017}.
These are annotated in the format of Dependency Grammar~\citep{hays1964dependency,hudson1984word,melcuk1988dependency,corbett1993heads,tesniere2015elements}.
In such dependency corpora, sentences are annotated with \emph{dependency trees} (Figure~\ref{fig:dependency}).
These are directed trees describing the grammatical relations among words. For example, the arcs labeled ``obj'' represent that the noun in question is the \emph{direct object} of the verb, rather than e.g. the subject or an indirect object.
A dependency arc is drawn from a \emph{head} (e.g. the verb `has') to a \emph{dependent} (e.g. its object `book').
Dependency trees can be defined in terms of many different syntactic theories \citep{corbett1993heads}.
Although there are some differences in how different formalisms would draw trees for certain sentences, there is broad enough agreement about dependency trees that it has been possible to develop large-scale dependency-annotated corpora of text from dozens of languages \citep{nivre2017universal}.

\begin{figure}
\centering
\begin{dependency}[theme = simple]
   \begin{deptext}[column sep=1em]
Mary \&	 has \& two \& green \& books  \\
   \end{deptext}
	%   \depedge[edge start x offset=-6pt]{2}{5}{ATT}
	%   \deproot{3}{ROOT}
   \depedge{2}{1}{nsubj}
   \depedge{2}{5}{obj}
   \depedge{5}{3}{nummod}
   \depedge{5}{4}{amod}
   %\depedge[arc angle=50]{7}{6}{ATT}
\end{dependency}
	\caption{An English sentence with dependency annotations, according to the Universal Dependencies 2.4 standard.
	We visualize grammatical relations as arcs drawn from heads (e.g., the verb `has') to dependents (e.g., its object `book').
	}\label{fig:dependency}
\end{figure}

%\paragraph{Selection of Languages}
We computed memory--surprisal tradeoffs for all languages for which there are Universal Dependencies 2.4 treebanks with a total of at least 500 sentences of training data.
We excluded data from historical languages, as these corpora often include poetry, translated text, or texts spanning several centuries.\footnote{Historical languages excluded: Ancient Greek, Classical Chinese, Coptic, Gothic, Latin, Old Church Slavonic, Old French.}
%\jd{add half a sentence on why?}
%While running this experiment, data from additional languages became available that also had enough data, through the Universal Dependencies 2.4 release. 
This resulted in 54 languages.

%\paragraph{Processing of Corpora}
For each of these languages, we pooled all available corpora into one dataset.
We excluded corpora that primarily contain code-switched text\footnote{Hindi English corpus} or text created by non-native speakers.\footnote{ESL for English, CFL for Chinese.}
Most Universal Dependencies corpora have a predefined split into \emph{training}, \emph{held-out} (also known as \emph{development}), and \emph{test} partitions.
%While larger corpora have all three partitions, smaller corpora often have only some of these partitions.
In most cases, we used the predefined data split, separately pooling data from the different partitions. 
For some languages with little data, there is no predefined training partition, or the training partition is smaller than the other partitions.
In these cases, we redefined the split to obtain more training data.
For these languages, we pooled all the available partitions, used 100 randomly selected sentences as held-out data, and used the remainder as training data.\footnote{This affects Amharic, Armenian, Breton, Buryat, Cantonese, Faroese, Kazakh, Kurmanji, Naija, Thai, and Uyghur.}
We did not make use of the \textit{test} partitions here.
We provide the sizes of the resulting datasets in SI Section 3.1.
The datasets ranged in size from 564 sentences (Armenian) to 114,304 sentences (Czech), with a median of 5,255 sentences per language.
For each language, we obtain a stationary process by concatenating the sentences from the corpus in random order, separated with an end-of-sentence symbol.


\subsection{Defining baselines}\label{sec:baselines}

Testing the Efficient Tradeoff Hypothesis requires comparing the memory--surprisal tradeoffs of real grammars to those of baseline grammars. The baseline grammars we construct are counterfactual ordering grammars that define consistent ordering rules similar to those found in actual languages (Figure~\ref{fig:grammars}).
For instance, these grammars specify which dependents precede or follow their heads (e.g., whether objects follow or precede verbs, whether adjectives follow or precede nouns), and the relative order of different dependents on the same side of the head (e.g., whether noun phrases have order adjective-numeral-noun or numeral-adjective-noun). Our formalism of ordering grammars was introduced in \citet{hahn2020universals}, adapting the method of \citet{gildea-optimizing-2007,gildea-grammars-2010,gildea-human-2015} to the setting of dependency corpora.


\begin{figure}
\centering
\includegraphics[width=\textwidth]{figures-gdrive/counterfactual-languages.pdf}
\caption{Estimating chance by constructing counterfactual grammars and languages: We start from an annotated dependency corpus of sentences annotated with syntactic dependencies (top left). We then extract the raw dependency structures, stripping away word order information (right left).
We construct baseline ordering grammars that provide rules for ordering the words in such dependency structures (Grammars 1--3).
Applying any such grammar to the dependency structures yields a counterfactual corpus of a hypothetical language that has the same dependency structures as the actual language, but different word order rules.}\label{fig:grammars}
\end{figure}


Universal Dependencies 2.4 defines 37 universal syntactic relations that are used to label dependency arcs across all corpora.
These relations encode cross-linguistically meaningful relations such as subjects (\textit{nsubj}, see Figure~\ref{fig:dependency}) , objects (\textit{obj}), and adjectival modifiers (\textit{amod}).
We define ordering grammars by assigning a parameter $a_\tau \in [-1,1]$ to every one of these 37 universal syntactic relations.
Relations sometimes have language-specific subtypes; we do not distinguish these subtypes.
Following Gildea and colleagues, this parameter defines how dependents are ordered relative to their head:
Given a head and a set of dependents, we order each dependent by the parameter $a_\tau$ assigned to the syntactic relation linking it to the head.
Dependents with negative weights are placed to the left of the head; dependents with positive weights are placed to the right. Ordering grammars describe languages that have consistent word order.
For instance, the subject is consistently ordered before or after the verb, depending on whether the parameter $a_{nsubj}$ for the verb-subject dependency is positive or negative.

We construct baseline grammars by randomly sampling the parameters $a_\tau$.
Such baseline grammars define languages that have consistent word order but, crucial for our purposes, do not exhibit systematic preferences for specific word order patterns such as short dependencies. %\jd{confusing rhetorical relation between previous and following sentence} 
%In actual languages, the ordering of words is largely determined by syntactic relations (CITE).


We first constructed at least 10 baseline grammars for each of the 54 real languages.
We then continued to construct baseline grammars until a precision-based stopping criterion was reached. This criterion was designed to ensure that enough grammars were sampled to reliably compare the tradeoff curves of real and baseline grammars, without biasing results towards or against our hypothesis (see SI Section 3.2.3).
The stopping criterion compared what fraction of baseline grammars had strictly more (or strictly less) efficient tradeoff curves than the real ordering, and required a bootstrapped 95\% confidence interval for that ratio to have width $\leq 0.15$.
The resulting number of baseline grammars ranged from 10 (Italian and Romanian) to 347 (Estonian).\footnote{Due to a scripting error, 846 grammars were generated for Erzya even though this was not required by the stopping criterion.}

Due to the way ordering grammars are specified, certain kinds of rules cannot be modeled by our word order grammars.
This includes rules sensitive to the category of the dependent, such as the difference between postverbal nominal objects and preverbal pronominal objects in Romance languages.
It also includes rules sensitive to larger context, e.g., the alternation between verb-final order in embedded clauses and verb-initial/verb-medial order in main clauses in German and Dutch.
Furthermore, the model does not allow rules specifying interactions between different constituents, for instance, verb-second order, where exactly one dependent precedes the verb, and all others follow it.
Finally, the model does not account for word order freedom, as all ordering choices are deterministic.
In this sense, ordering grammars only represent an approximation to the kinds of ordering rules found in natural language \citep{gildea-optimizing-2007, gildea-grammars-2010, gildea-human-2015}.
Other models described in the literature \citep{futrell2015experiments, wang2016galactic} mostly share these limitations.
See Sections~\ref{subsec:expt2-discussion} and \ref{subsec:freedom} for further discussion and control studies addressing these limitations.
%\jd{this paragraph is a little confusing. is this supposed to convince the reader that ordering grammars are a good formalism within which to construct baseline grammars? if so, pre-empt the question about the effect this will have for free word order languages by explicitly mentioning the issue and deferring its discussion to later, rather than simply saying that word order freedom is not modeled (about which the uninitiated reader will wonder whether that's a good or a bad thing)}


To ensure that results are not due to the representational restrictions of the word order grammar formalism, we also compare the real languages to the result of ordering the corpora according to grammars that approximates the observed orders to the extent possible in the grammar formalism.
These grammars have exactly the same representational constraints as the baseline grammars.
Importantly, they differ from real languages in having entirely deterministic order.
We expect these grammars to have better memory-surprisal tradeoffs than comparable random baseline grammars across all languages.

We create ordering grammars that are fit to the actual orderings of each language.
These grammars faithfully represent the ordering rules of the actual language, to the extent  possible in the formalism of ordering grammars:
They match the order of the actual language in those cases where order of a relation is fully consistent; for relations where order is variable, they approximate this by modeling the most frequent order.
In representing word order rules, they have the same limitations as baseline grammars have, for instance, they cannot specify rules sensitive to the category of the dependent or to larger context.
These grammars are extracted from the observed orders using the method of \cite{hahn2020universals}.


\subsection{Results}\label{sec:main-experiment-results}
To test the Efficient Tradeoff Curve Hypothesis, we compare the tradeoff curves for the real orders with those for random baseline grammars.
In Figure~\ref{fig:it}, we show the estimated values of $I_t$ for real orders (blue) and the median of $I_t$ across different baseline grammars (red).
In most languages, $I_1$ is distinctly larger for the actual orderings (blue) compared to the baseline orderings (red). This means that real orderings tend to concentrate more predictive information at the immediately preceding word than baseline grammars.

In Figure~\ref{fig:median-table-expt2}, we show the resulting bounds on the memory-surprisal tradeoff curves, showing median surprisals at given levels of memory, for real and baseline languages.
We compute surprisal at 40 evenly spaced points of memory (selected individually for each language, between 0 and the maximal memory capacity $H_M$ obtained using Theorem~\ref{prop:suboptimal}), over real orders and baseline grammars.
We then compute the median surprisal over all runs for the real language, and over all baselines grammars.
For each point, we compute an non-asymptotic and non-parametric 95\% confidence interval for median surprisal using the binomial test.
%using the binomial PDF. \mhahn{some standard stats reference}
%This is an \emph{exact} confidence interval, without parametric assumptions or asymptotic approximations.

Numerically, the real language provides a better tradeoff than the median of the baselines across all languages, with four exceptions (Latvian, North Sami, Polish, Slovak). In order to quantify the degree of optimality of real orders, we further computed the area under the memory-surprisal tradeoff curve (AUC) for real and baseline orderings.
Area under the curve (AUC) is a general quantity evaluating the efficiency of a tradeoff curve \citep{bradley1997use}.
A \emph{smaller} area indicates a \emph{more efficient} memory-surprisal tradeoff.
In Figure~\ref{fig:auc}, we plot the AUC for the real orderings, together with the distribution of AUCs for baseline grammars.
We quantify the degree of optimality by the fraction of baseline grammars for which the AUC is higher than for the real orders:
The real ordering is highly efficient if it results in a lower AUC than almost all baseline grammars.
Numerically, the AUC is smaller in the real orderings than in at least 50\% of baseline grammars in all but three languages (Polish, Slovak, North Sami).
We evaluated significance using a two-sided binomial test.
In these three languages, the AUC is higher in the real orderings than in significantly less than 50\% ($p < 0.01$ in each language).
In all other languages except for Latvian, the fraction of more efficient baseline grammars was significantly less than 50\%, at $p=0.01$, where we applied Hochberg's step-up procedure \citep{hochberg1988sharper} to control for multiple comparisons.
In 42 of the 54 languages, the real language was more efficient than all of the sampled baseline grammars. % fraction of more efficient baseline grammars out of the samples taken was 0\%.



The AUC for the fitted grammars is lower than more than 50\% of random baseline grammars in all 54 languages ($p < 0.01$, using two-sided Binomial test and Hochberg's step-up procedure). Thus, we replicate the result that ordering regularities of real languages provide more efficient tradeoffs than most possible order grammars even when comparing within the same word order grammar formalism.


\begin{figure}
	\begin{center}
\includegraphics[width=\textwidth]{it-table-mle.pdf}
\end{center}
	\caption{Conditional mutual information $I_t$ (y-axis) as a function of $t$ (x-axis), for real (blue), fitted (red) and baseline (green) orders. We plot the median over all sampled baseline grammars.}\label{fig:it}
\end{figure}



\begin{figure}
	\begin{center}
\includegraphics[width=\textwidth]{results-table-mle.pdf}
\end{center}
	\caption{Surprisal (y-axis) at given memory level (x-axis), for real (blue), fitted (red), and baseline (green) orders.
	For the real (blue) and fitted (red) orders, we provide the median across multiple random seeds of the neural network estimator for $I_t$ (see SI Section 3.2.2), and 95\% confidence bands.
	For the baseline grammars (green), we provide the median across both the sampled baseline grammars and multiple random seeds of the estimator, and 95\% confidence bands for this median.
}\label{fig:median-table-expt2}
\end{figure}


\begin{figure}
	\begin{center}
\includegraphics[width=\textwidth]{auc-table_MLE.pdf}
\end{center}
\caption{Histograms for the Area under the Curve (AUC) for the memory--surprisal tradeoffs for real (blue), fitted (red), and random (green) orders.
We provide a kernel density smoothing estimate of the distribution of random baseline orders.
A smaller AUC value indicates a more efficient tradeoff.
In most cases, the real and fitted orders provide more efficient tradeoffs than most or all baseline grammars.
}\label{fig:auc}
\end{figure}




\subsection{Discussion}\label{subsec:expt2-discussion}

We have found that 50 out of 54 languages provide better memory-surprisal tradeoffs than random baselines with consistent but counterfactual word order rules.
Four languages provide exceptions; these are Latvian (Baltic), North Sami (Uralic), Polish and Slovak (both Slavic); these four languages did not have significantly lower AUC values than half of the random baselines.
One feature that unites these four languages is that they have strong word order freedom, as we will see below in Figure~\ref{fig:freedom-surp}. % \mhahn{(also CITE something)}.
Word order freedom plausibly makes sentences less predictable, as the same syntactic structure can receive different surface realizations.
We thus hypothesized that word order freedom  impacts the memory-surprisal tradeoff, and that languages with more strongly fixed word order should display more optimal memory-surprisal tradeoffs.


To test this hypothesis, we examined the correlation between word order freedom and the surprisal difference between real and baseline orderings.
To quantify word order freedom, we used a corpus-based estimate, the \key{branching direction entropy}~\citep{futrell-quantifying-2015}.
This is the entropy of the ordering (head-first or dependent-first) of dependencies conditioned on the dependency label and the part-of-speech label of head and dependent.
These two quantities are plotted in Figure~\ref{fig:freedom-surp}.
We found that branching direction entropy was strongly correlated with the surprisal difference between real and baseline orderings (Spearman correlation $-0.58$, $p < .0001$).

This result might mean that optimization of word orders for memory--surprisal tradeoffs is indeed stronger in languages with more fixed word order, and that word order freedom leads to less efficient memory--surprisal tradeoffs.
A second possibility is that languages with seemingly free word order encode other information in word order, in particular, information about information structure \citep[e.g.][]{givon1988pragmatics,firbas1966defining,firbas1974aspects,myhill1985pragmatic}.
In Section~\ref{subsec:freedom}, we will test the latter hypothesis by examining whether the degree of optimization changes when taking into account information structure.




\begin{figure}
\includegraphics[width=0.95\textwidth]{figures/surprisal-branching-entropy-REAL-invert.pdf}
	\caption{Word order freedom and strength of optimization: For each of the 54 languages, we show word order freedom as measured by branching entropy, and the difference between the real order's surprisal and the mean surprisal of the baseline grammars, at the maximum memory value (see Figure~\ref{fig:median-table-expt2}).
	Languages with higher branching direction entropy show a smaller reduction in surprisal compared to baseline orders.%\jd{is there sth to say about the outliers -- uyghur, arabic, french?}
	}\label{fig:freedom-surp}
\end{figure}


%\mhahn{One thing that has to be discussed: the absolute values differ between languages}



%Relatedly, a possible concern with this experiment is that the optimization observed in Study 2 might be an artifact of the grammar formalism:
%The baseline grammars, by virtue of being ordering grammars, cannot exactly represent all word order rules found in real languages, and this restriction in representational capacity might potentially impact the measured optimality of the memory-surprisal tradeoffs.
%We will address this in Section~\ref{subsec:freedom} by comparing the baseline grammars against grammars that represent the word order of the real languages as faithfully as possible within the grammar formalism.




\subsection{Controlling for Information Structure}\label{subsec:freedom}


%We test this hypothesis by comparing baseline languages to \emph{fixed-order} versions of the real languages.
%This enables us to tease apart the impact of the languages' word order rules from the impact of word order freedom.



We now address points raised above by 



\begin{enumerate}
\item First, to ensure that results are not due to the representational restrictions of the word order grammar formalism, we compare the real languages to the result of ordering the corpora according to grammars that approximates the observed orders to the extent possible in the grammar formalism.
These grammars have exactly the same representational constraints as the baseline grammars.
Importantly, they differ from real languages in having entirely deterministic order.
We expect these grammars to have better memory-surprisal tradeoffs than comparable random baseline grammars across all languages.

\item 
Languages with flexible word order often show a strong influence of information structure on word order.
Only relatively few datasets have annotations for information structure, and even fewer datasets have both syntactic and information structure annotation.
We draw on the Prague Dependency Treebank of Czech, which has both types of annotation.
Czech is a language with relatively high degree of word order freedom.

Simulation taking information structure into account (for one language).
Due to the difficulty of annotating information structure, few such corpora are available.
We use the Prague Dependency Treebank, which has information structure annotation.
\end{enumerate}




\subsection{Setup}

Everything is identical to Experiment 1, except that we replaced the real orderings with orderings generated from grammars fitted to the word orders found in the real language.

\paragraph{Fitting Ordering Grammars to Actual Orders}
We create ordering grammars that are fit to the actual orderings of each language.
These grammars faithfully represent the ordering rules if the actual language, to the extent that is possible in the formalism of ordering grammars.

We construct these grammars by constructing \emph{probabilistic ordering grammars}, and setting the parameters to maximize the \emph{likelihood} of the actually observed orderings.
The method is fully taken from \cite{hahn-universals-2020}.
%We parameterized probabilistic ordering grammars as follows.
%For each relation type $\tau$, we introduce a \emph{direction parameter} $a_\tau \in [0,1]$ and a \emph{distance parameter} $b_\tau \in \mathbb{R}$.
%Each dependent is ordered on the left of its head with probability $a_\tau$ and to the right with probability $1-a_\tau$. 
%Then for each set of co-dependents $\{s_1, \dots , s_n\}$ placed on one side of a head, their order outward from the head is determined by iteratively sampling from the distribution $\operatorname{softmax}(b_{\tau_1}, \dots, b_{\tau_n})$ (\cite{goodfellow2016deep}, p. 184) without replacement. 
%Given a dependency tree, a probabilistic ordering grammar assigns a probability distribution over the possible projective linearizations of that tree.
%We use gradient descent to find parameters $a_\tau, b_\tau$ so as to maximize the overall likelihood of the orders in the actual corpus.
%We convert probabilistic ordering grammars into ordinary ordering grammars by the following method.
%Let $A_-$ be those relations $\tau$ where $a_\tau > 0.5$, similarly for $A_+$ those here $a_\tau \geq 0.5$.
%Then we order all relations in $A_-$ by $b_\tau$ in \emph{decreasing} order, and those in $A_+$ by $b_\tau$ in \emph{increasing} order.
%Then ordering a tree following the converted version is equivalent to greedily choosing the highest-probability linearization for the dependents of each head in a tree.
%We choose this method since maximum-likelihood grammars can be constructed with simple gradient descent.
%Another option would be to use some kind of discrete optimization method to approximate the original orders without a probabilistic method.
%However, discrete optimization is computationally challenging.


\paragraph{Taking Information Structure into Account}

Part of the the Prague Dependency Treebank has annotation for topic-focus articulation \cite{mikulova2006annotation}.
Constituents are annotated for contrastiveness and for contextual boundedness, i.e., givenness.
Three labels are used:
``c'' for contrastive and contextually bound, ``f'' for contextually non-bound, ``t'' for non-contrastive contextually bound.
These labels are diagnosed based on constituent order and intonation.
Some constituents remain unmarked, the vast majority of these are function words such as adpositions, conjunctions, and auxiliaries; we introduce a label ``NA'' for these.

We extend the word order grammar formalism by defining separate weights for each combination of the 37 syntactic relations and these four information structure labels.

We obtained 38,727 training sentences and 5,228 dev sentences.


\subsection{Results}

\paragraph{Fitted Grammars}

We show estimated tradeoffs in Figure (XX).

In X languages, the fitted grammars provide better tradeoffs than at least 50\% of the random grammars.

The AUC is lower than more than 50\% of random baseline grammars in all 54 languages ($p < 0.01$, using two-sided Binomial test and Hochberg's step-up procedure).


\begin{figure}
	\begin{center}
\includegraphics[width=\textwidth]{results-table-mle.pdf}
\end{center}
	\caption{Median surprisal (y-axis) at given memory level (x-axis), for real orders (blue) and random baseline grammars (red). We provide 95\% confidence bands. These are computed over different runs of the estimation algorithm for the real orders, and over different runs \emph{and} different grammars for the baseline grammars.}\label{fig:median-table}
\end{figure}



%\begin{figure}
%\includegraphics[width=0.5\textwidth]{figures/full-GROUND-listener-surprisal-memory-HIST_z_byMem_onlyWordForms_boundedVocab.pdf}
%	\caption{Histogram}\label{fig:hist-real}
%\end{figure}

\paragraph{Information Structure}

We show estimated tradeoffs in Figure~\ref{fig:median-czech-infostruc}.
Note that this experiment was conducted only on the subset of the Prague Dependency Treebank that has information structure annotation; thus, the numerical values are slightly different from those in Section X.




\begin{figure}
\includegraphics[width=0.5\textwidth]{figures/Czech-PDT-listener-surprisal-memory-MEDIANS_onlyWordForms_boundedVocab.pdf}
	\caption{Czech with information structure. Green: Baselines with information structure. Red: Baselines without information structure. Blue: Real}\label{fig:median-czech-infostruc}
\end{figure}



\begin{figure}
\includegraphics[width=0.9\textwidth]{figures/surprisal-branching-entropy-REAL-infostruc.pdf}
	\caption{Order Freedom vs Difference in Surprisal at maximal memory (compare Figure (REF)). The arrow indicates how the data point for Czech would move when modeling word order including information structure.}\label{fig:freedom-mi-with-infostruc}
\end{figure}




%
%\subsection{Discussion: Alternative Models}
%In view of the NLP literature, the following are the main other options that exist for estimating mutual information and probabilities in sequences:
%
%A traditional model uses n-gram models. A challenge of n-gram models is that they do not express any morphosyntactic generalizations. Furthermore, standard n-gram models do not express any generalizations about pairs of words that are not adjacent -- e.g., encoding a generalization about morphological agreement between two words is hard for such a model to capture if the two words are not always adjacent. Both the small scale of available corpora in many languages and free word order in many languages with rich morphology thus seem to make such models unattractive.
%We evaluate our hypothesis using n-gram models in SI Section X, confirming the conclusions obtained from neural models.
%
%A second option is to construct a statistical grammar, such as PCFG.
%The challenge is to encode statistical morphosyntactic generalizations, and to decide which independence assumptions to put into the model.
%One can either decide on a language-specific basis which generalizations to put in (laborious and might introduce bias), or choose a general model family that is rich enough to learn generalizations.
%The second option will make this a machine learning model that, for our purposes, does not seem to be superior to a recurrent neural network.
%



%\subsection{Data}
%\subsection{Setup}
%The recurrent neural network architecture has a range of adjustable parameters such as the number of neurons.
%For each language, we used Bayesian optimization using the Expected Improvement acquisition function (CITE) \citep{snoek-practical-2012} to find a good setting of the hyperparameters, taking average surprisal on random grammars as the objective.
%This biases the hyperparameters towards favoring counterfactual grammars.

%\subsection{Setup}




%\paragraph{Data}
%Given a sequence of input words $w_1, ..., w_n \in V$, the model 
%%
%\textbf{TODO I'm describing this in a lot of detail. Alternatively, we can say this is a standard NLP method and refer to the NLP literature for the definition.}
%The first component of such a model is an \emph{embedding matrix} $W_{emb} \in \mathbb{R}^{|V| \times d_{emb}}$, where the \emph{vocabulary} $\mathcal{V}$ is a set, containing the words that occur in the corpus, and $d_{emb} \in \mathbb{N}$ is a fixed parameter.
%This matrix assigns a $d_{emb}$-dimensional vector to each word occurring in the corpus.
%The second component is an LSTM cell $f_{LSTM}$, a nonlinear transformation mapping an \emph{input} vector $x_{i} \in \mathbb{R}^{d_{emb}}$ a \emph{hidden state} $h_i \in \mathbb{R}^{d_{LSTM}}$ and a \emph{cell state} $c_i \in \mathbb{R}^{d_{LSTM}}$ to a new pair of hidden state and cell states $h_{i+1}, c_{i+1} \in \mathbb{R}^{d_{LSTM}}$.
%The LSTM cell $f_{LSTM}$ is parameterized by a matrix of numerical parameters $W_{LSTM}$.
%
%%Such networks estimate the probability of a word in context as follows.
%Given a sequence of input words $w_1, ..., w_n \in V$, the model first retrieves fixed-dimensionality vector representations $x_1, ..., x_n$, where $x_i$ is the row of $W_{emb}$ corresponding to the word $w_i$.
%It then computes a sequence of hidden and cell states by the following recurrent computation:
%\begin{align*}
%	h_1, c_1 &:= 0 \\
%	h_2, c_2 &:= f_{LSTM}(x_1, h_1, c_1) \\
%	\dots \\
%	h_{n+1}, c_{n+1} &:= f_{LSTM}(x_n, h_n, c_n) \\
%\end{align*}
%The vector $h_i$ encodes the result of reading the words $w_1, ..., w_{i-1}$.
%We will write $LSTM(w_1, ..., w_{i-1})$ for $h_i$.
%
%The third component of the recurrent language model is the matrix $W_{output} \in \mathbb{R}^{|V| \times d_{LSTM}}$.
%We obtain per-word predictions of the next word by computing
%\begin{align*}
%	s_i := W_{output} h_i \in \mathbb{R}^{|V|} \\
%	p_i := \operatorname{softmax}(s_i)\in \mathbb{R}^{|V|} 
%\end{align*}
%where the softmax transformation normalizes vectors into probability distributions as follows
%\begin{equation}
%	\operatorname{softmax}(x)_i := \frac{\exp(x_i)}{\sum_{j=1}^{|V|} \exp(x_j)}
%\end{equation}
%Finally, the probability of the word $w_n$ in the context $w_1, ..., w_{n-1}$ is computed as
%\begin{equation}
%	p_\theta(w_n|w_1...w_{n-1}) := \frac{\exp((p_n)_{w_n})}{\sum_{w \in V} \exp(x_w)}
%\end{equation}
%and thus the surprisal is estimated as
%\begin{equation}
%- \log	p_\theta(w_n|w_1...w_{n-1}) := -\log \frac{\exp((p_n)_{w_n})}{\sum_{w \in V} \exp(x_w)}
%\end{equation}
%We discuss the choice of the numerical parameters in the next section.
%



%We collected data from the actual and random orderings in proportion one to two.
%The stopping criterion will be described below.

%Due to the randomness both in the sequence of training examples and the random initialization of the network weights, the results of the parameter estimation procedure will vary when run multiple times, especially on smaller datasets.
%Informally, due to the finiteness of the dataset, multiple parameter settings are compatible with the available training data.
%Consequently, memory-surprisal tradeoffs estimated on held-out sets will also show some variation.
%Therefore, we collect multiple samples for the actual orderings to control for variation due to the random initialization of the neural network.


%We chose these thresholds based on preliminary simulations which had suggested that these widths were achievable at acceptable computational cost.

%- at least 30 samples from both baseline and real
%
%- for the language-level tradeoff curve, either the fraction is zero or the bootstrapped CI has width $\leq 0.2$.



%
%(1) is bigram MI always greater in real languages?
%
%(2) is the tradeoff curve always lower than for deterministic simple grammar? for deterministic complex grammars? for stochastic simple/complex grammars?




%Training progresses in a series of parameter update steps, constructing updated parameters $\theta_0, \theta_1, \theta_2, \dots$.
%In the $n$-th update step, we first randomly select a word sequence $w_1 ... w_T$ from the training corpus, and use the LSTM using the current parameter setting $\theta_n$ to compute the per-word surprisals.
%We then update the parameter vector:
%\mhahn{maybe better to just say we use SGD}
%\begin{equation}\label{eq:train}
%	\theta_{n+1} := \theta_n + \alpha \partial_\theta \left(\sum_{i=1}^T \log p_\theta(w_i|w_1...w_{i-1})\right)
%\end{equation}
%where $\alpha \in \mathbb{R}_+$ is the \emph{learning rate}.



\section{Study 3: Morpheme Order}\label{sec:morphemes}

%- datasets
%-- unimorph
%-- bibles corpus?
%-- celex
%- permute phonemes
%- permute syllables
%- permute phonemes only within syllables
%- permute morphemes 
%\section{Morphology}

The memory--surprisal trade-off described in Section~\ref{sec:ms-tradeoff} applies not only at the level of words, but also at the level of any linguistic element, including morphemes within words. Therefore, just as word orders are influenced by information locality, the order of morphemes within words should be structured so that morphemes which predict each other are close to each other.

Here, we apply the memory--surprisal trade-off to explain the order of morphemes within morphologically complex words in two agglutinative languages. We study two agglutinative languages for which extensive corpora with hand-annotated morphological segmentation and labeling are available: Japanese and Sesotho. Below, we give brief sketches of the morphological patterns in these languages.


\paragraph{Verb Suffixes in Japanese}

In Japanese, verbs are marked with an extensive number of suffixes. For example, the following verb forms are marked with multiple suffixes:

\ex.\ag. mi  rare mash yoo \\
%Stem (3) (5) (8) \\
see  \textsc{passive}  \textsc{politeness}  \textsc{future} \\
`will be seen'
\bg. mi taku nakat ta \\
%Stem (6) (7) (8) \\
see \textsc{desiderative} \textsc{negation} \textsc{past} \\
`did not wish to see'

Based on corpus data and the linguistic literature on Japanese, we identified the following frequent verb suffixes, occurring in the following order outwards from the verb root (see SI for details).


\begin{enumerate}
\item \textit{suru}: obligatory suffix after Sino-Japanese words when they are used as verbs
\item Valence: causative (-\textit{ase}-) (\citet[142]{hasegawa2014japanese}, \citet[Chapter 13]{kaiser2013japanese})
\item Voice: passive (-\textit{are}-, -\textit{rare}-) \cite[152]{hasegawa2014japanese} \cite[Chapter 12]{kaiser2013japanese}
\item Mood: potential (-\textit{e}-, -\textit{are}-, -\textit{rare}-) \cite[398]{kaiser2013japanese}  
\item Politeness (-\textit{mas}-) \cite[190]{kaiser2013japanese}.
\item Mood: desiderative (-\textit{ta}-) \cite[238]{kaiser2013japanese}
\item Negation (-\textit{n}-)
\item Tense/Aspect/Mood/Finiteness: past (-\textit{ta}), future/hortative (-\textit{yoo}) \cite[229]{kaiser2013japanese}, nonfiniteness (-\textit{te}) \cite[186]{kaiser2013japanese}
%\item Nonfiniteness: the suffix -\textit{te} derives a nonfinite form .
\end{enumerate}

%In accordance with Bybee's hierarchy, valence is marked closest to the verb, followed by voice.
%Unlike predicted by the hierarchy, tense/aspect markers are not placed closer to the verb than mood/modality markers.






\paragraph{Verb Affixes in Sesotho}
Sesotho (also known as Southern Sotho) is a Southern Bantu language, spoken by 5.6 million L1 speakers (REF) mainly in Lesotho and South Africa.
Sesotho verbs are marked with both prefixes and suffixes.
Common prefixes include markers for agreement with subjects and objects; object prefixes always follow subject prefixes \ref{ex:oadireka}.
Common suffixes include markers changing valence and voice, and a mood suffix \ref{ex:ophehela}.

\ex.\ag. oa di rek a \\
\textsc{subject.agreement} \textsc{object.agreement} buy \textsc{indicative} \\
`(he) is buying (it)'  \cite[ex (41)]{demuth1992acquisition} \label{ex:oadireka}
\bg. o pheh el a \\
\textsc{subject.agreement} cook \textsc{applicative} \textsc{indicative} \\
`(he) cooks (food) for (him)'  \cite[ex (41)]{demuth1992acquisition}
\label{ex:ophehela}

We identified affix morphemes and their ordering based on the analysis in \cite{demuth1992acquisition}, supplemented with information from grammars of Sesotho \citep{doke1967textbook,guma1971outline}. See SI for details.
We identified the following prefixes:

\begin{enumerate}
    \item Subject agreement: This morpheme encodes agreement with the subject, for person, number, and noun class (the latter only in the 3rd person) \cite[\textsection 395]{doke1967textbook}.
	    The annotation provided by \cite{demuth1992acquisition} distinguishes between ordinary subject agreement prefixes and agreement prefixes used in relative clauses; we distinguish these morpheme types here.
    
    \item Negation \cite[\textsection 429]{doke1967textbook}
    
    \item Tense/aspect marker   \cite[\textsection 400--424]{doke1967textbook}
    
    \item Object agreement or reflexive marker \cite[\textsection 459]{doke1967textbook}. 
    Similar to subject agreement, object agreement denotes person, number, and noun class features of the object.
\end{enumerate}
We identified the following suffixes:

% something we might cite at some point (pointers from Beth Levine), about templatic morpheme order in Bantu
% Hyman2003 in literature.bib
% Jeffrey Good, Strong Linearity: Three Case Studies Towards a Theory of Morphosyntactic Templatic Constructions (Diss, 2003)
    
\begin{enumerate}
\item Semantic derivation: reversive (e.g., do $\rightarrow$ undo)  \cite[\textsection 345]{doke1967textbook}
\item Valence: Common valence-altering suffixes include causative, neuter/stative, applicative, and reciprocal \cite[\textsection 307--338]{doke1967textbook}. See SI for details on their meanings.
    \item Voice: passive \cite[\textsection 300]{doke1967textbook} 
    \item Tense \cite[\textsection 369]{doke1967textbook}
    \item Mood \cite[\textsection 386--445]{doke1967textbook}
    \item Interrogative and relative markers, appended to verbs in certain interrogative and relative clauses \cite[\textsection 160, 271, 320, 714, 793]{doke1967textbook}.
    %-\textit{ng}.
    %The interrogative marker -\textit{ng} is an enclitized form of the interrogative `what'.
    %The relative marker -\textit{ng} is suffixed to verbs in relative clauses.
\end{enumerate}



%For prefixes, in agreement with Bybee's hierarchy, subject agreement is encoded in a position further away than TAM features.
%For suffixes, relation to Bybee hierarchy: valence closest, then voice, then tense, then mood.


\subsection{Experiment}
\paragraph{Data Selection and Processing}
For Japanese, we drew on Universal Dependencies data.
In the tokenization scheme used for Japanese, most affixes are separated as individual tokens, effectively providing morpheme segmentations.
We used the GSD corpus, Version 2.4, \citep{tanaka2016universal, asahara2018universal}, as it was the only corpus with a training set and freely available word forms.
In the corpus, verb suffixes largely correspond to auxiliaries (with tag \texttt{AUX}); only a few morphemes tagged \texttt{AUX} are not standardly treated as suffixes (see SI), and one frequent suffix (-\textit{te}) is labeled \texttt{SCONJ}.
We selected verb forms by selecting all chains of a verb (tag \texttt{VERB}) followed by any number of auxiliaries (tag \texttt{AUX}) from the training set of the corpus. When the suffix -\textit{te} (tag \texttt{SCONJ}) followed such a chain, we added this.
We labeled suffixes for underlying morphemes with the help of the lemmatization provided for each suffix in the corpus (see SI for details).
We obtained 15,281 verb forms in the training set and 1,048 verb forms in the held-out set.
Of the forms in the training set, 27\% had two or more suffixes (modal group: two suffixes, 20\%; maximum seven suffixes).
While predicting order naturally focuses on datapoints with more than one suffix, we include the other datapoints for estimating conditional mutual information $I_t$.
%We used the annotation provided in the corpus to identify underlying morphemes (see SI).

%In the corpus, each morpheme is annotated with a lemma, indicating a normalized context-independent representation of the morpheme.
%For both verbs and affixes, this lemma annotation abstracts away from most morphophonological and other allomorphic variation, it thus largely identifies underlying morphemes (see SI for limitations).

For Sesotho, we used the Demuth Corpus \citep{demuth1992acquisition} of child and child-directed speech, containing about 13K utterances with 500K morphemes.
The corpus has very extensive manual morphological segmentation and annotation; each verb form is segmented into morphemes, which are annotated for their function.
Sesotho verbs carry both prefixes and suffixes.
We extracted 37K verb forms (see SI for details).
We randomly selected 5\% to serve as held-out data and used the remaining 95\% as training data.
93\% of forms had two or more affixes (modal group: three affixes, 36\%; maximum eight affixes).

\paragraph{Estimating Memory-Surprisal Tradeoff}
As in many languages, affixes in Japanese and Sesotho show morphophonemic interactions between neighboring morphemes; for instance, the Japanese politeness morpheme -\textit{mas}- takes the form -\textit{masu} when it is word-final, while it has the allomorph -\textit{mase} when followed by the negation suffix -\textit{n}.

To control for these interactions, we model prediction on the level of morphemes sequences.
That is, we represent each verb form as a sequence of a stem and suffix morphemes, abstracting away from morphophonemic interactions.
See SI for results when modeling prediction on the level of phonemes; these results come out qualitatively similar to the results we present here.

We calculate $I_t$ by estimating an $n$-gram model on the training set and then computing the average surprisal $S_t$ as cross-entropy on the held-out set using Kneser-Ney smoothing.
The model may overfit as the context size $t$ increases, leading to higher cross-entropies for larger values of $t$.
To mitigate this problem, we estimate
\begin{equation}
\hat{S}_t := \min_{s \leq t} S_s,
\end{equation}
where $S_s$ is the cross-entropy of the $s$'th order Markov model on held-out data.
This procedure ensures that $\hat{S}_t$ can only decrease as the context size $t$ increases.
Similar results are obtained when instead using the simple `naive' estimator for $I_t$ on the training set; see the SI for details.

\paragraph{Parameterizing Alternative Orderings}

We parameterize alternative affix orderings by assigning a weight in $[0,1]$ to each morpheme.
Given such a grammar, affixes are ordered by the values assigned to their underlying morphemes.
We consider all morphemes annotated in the corpora, including low-frequency ones going beyond the ones identified above (see SI for details).

In Japanese, the passive (slot 3) and potential (slot 4) markers are formally indistinguishable for many verbs.
As we cannot systematically distinguish them on the basis of the available corpus annotation, we merge these into a single underlying morpheme `Passive/Potential'.


To verify that this formalism is appropriate to capturing moroheme order in Japanese and Sesotho, we also fitted models parameterized in this way to the observed orders.
We found that ordering morphemes according to these fitted models recovered the observed order for almost all forms (98.6 \% for Japanese, 99.93\% for Sesotho prefixes, 97.4\% for Sesotho suffixes).
Exceptions largely concern low-frequency suffixes beyond those considered here.


\paragraph{Creating Optimal Orders}

We optimize orderings for the AUC under the memory-surprisal tradeoff curve with an adaptation of the hill climbing method used by \citet{gildea-human-2015} to optimize word order grammars for the length of syntactic dependencies and trigram surprisal.

We randomly initialize the assignment of weights to morphemes, and then iteratively change the assignment to reduce AUC.
In each iteration, we randomly choose one morpheme, and evaluate AUC for each way of ordering it between two other morphemes.
We then update the weights to the ordering that yields the lowest AUC.
To speed up optimization, we restrict to morphemes occurring at least 10 times in the corpus for 95\% of iterations, and to 10\% of possible orderings in each step.
These choices vastly reduce computation time by reducing time spent on low-frequency morphemes.
This optimization method is approximate, as it only guarantees convergence to a local optimum, not to a globally optimal assignment.

We ran this method for 1,000 iterations. Empirically, AUC converges after a few hundred iterations.
To control for the randomness in initialization and the optimization steps, we ran this algorithm ten times.
For Sesotho, we ran the algorithm separetely for prefixes and suffixes.

\subsection{Results}
In Figure~\ref{fig:morph-auc}, we compare the area under curve of the memory--surprisal trade-off for Japanese and Sesotho under different orderings.
Both observed orders and the approximately optimized grammars show lower AUCs than all random baseline samples.
For comparison, we also show AUC for the order resulting from \emph{reversing} all suffix chains in the observed orders; this results in high AUC even exceeding most random grammars.
These results show that Japanese and Sesotho affix orderings enable approximately optimal memory-surprisal tradeoffs.



\begin{figure}
	\begin{center}
\includegraphics[width=\textwidth]{figures/Both-suffixes-byMorphemes-auc-hist-heldout.pdf}
\end{center}
	\caption{Areas under the curve for the memory-surprisal tradeoff for verb affixes in Japanese (left) and Sesotho (right). \mhahn{make numbers larger}}
	\label{fig:morph-auc}
\end{figure}



%\begin{figure}
%	\begin{center}
%\includegraphics[width=0.3\textwidth]{figures/Japanese-suffixes-byPhonemes-it-heldout.pdf}
%\includegraphics[width=0.3\textwidth]{figures/Japanese-suffixes-byPhonemes-memsurp-heldout.pdf}
%\includegraphics[width=0.3\textwidth]{figures/Japanese-suffixes-byPhonemes-auc-hist-heldout.pdf}
%
%\includegraphics[width=0.3\textwidth]{figures/Japanese-suffixes-byMorphemes-it-heldout.pdf}
%\includegraphics[width=0.3\textwidth]{figures/Japanese-suffixes-byMorphemes-memsurp-heldout.pdf}
%\includegraphics[width=0.3\textwidth]{figures/Japanese-suffixes-byMorphemes-auc-hist-heldout.pdf}
%\end{center}
%	\caption{Areas under the memory-surprisal tradeoff curve for Japanese verb suffixes.}\label{fig:jap-phon-morph}
	%\caption{Japanese verb suffixes, measuring prediction on the level of phonemes (top) and morphemes (bottom), for real (blue), random (green), and approximately optimized (red) orderings. Left: $I_t$ as a function of $t$. Center: Memory-surprisal tradeoff. Right: Areas under the curve for the memory-surprisal tradeoff.}\label{fig:jap-phon-morph}
%\end{figure}


We now ask to what extent the observed morpheme ordering is predicted correctly by approximately optimized grammars.

In Table~\ref{tab:morph-acc}, we give summary statistics about the accuracy of optimized grammars in predicting affix order in the corpus, together with random baseline figures.
We evaluate accuracy using two methods:
In one method (`Pairs'), we consider, for each verb form in the corpus, all pairs of prefixes (or suffixes).
We then evaluate for what fraction of these pairs, across the corpus, the relative order of the two affixes is as predicted by the grammar.
In the other method, (`Full'), we count what fraction of verb forms in the corpus has exactly the affix ordering predicted by the grammar.

\begin{table}
\begin{tabular}{cc||ll|ll}
             &              & \multicolumn{2}{c}{Prefixes}    & \multicolumn{2}{|c}{Suffixes} \\
             &              & Pairs & Full & Pairs & Full \\ \hline\hline
Japanese & Optimized  & -- &  -- &   0.954 (SD 0.009) & 0.945 (SD 0.012) \\ 
& Baseline    & -- & -- & 0.504 (SD 0.0) & 0.414 (SD 0.0) \\ \hline
%Phon.   &   Optimized  &  0.993 (SD 0.0) & 0.989 (SD 0.0) & 0.792 (SD 0.102) & 0.734 (SD 0.13) \\
%	& Random  &  0.322 (SD 0.253) & 0.195 (SD 0.24) & 0.588 (SD 0.155) & 0.554 (SD 0.167) \\ \hline
Sesotho &   Optimized  &  0.988 (SD 0.0) & 0.989 (SD 0.0) & 0.756 (SD 0.014) & 0.676 (SD 0.017) \\
&   Baseline  &  0.672 (SD 0.305) & 0.604 (SD 0.338) & 0.423 (SD 0.204) & 0.332 (SD 0.211) \\ 
\end{tabular}
\caption{Accuracy of approximately optimized orderings, and of random baseline orderings, in predicting verb affix order in Japanese and Sesotho. `Pairs' denotes the rate of pairs of morphemes that are ordered correctly, and `Full' denotes the rate of verb forms where order is predicted entirely correctly. We show means and standard deviations over different runs of the optimization algorithm (`Optimized'), and over different random orderings (`Random').}\label{tab:morph-acc}
\end{table}

\paragraph{Japanese results.} In Japanese, by both measures, optimized grammars recover the observed orders with high accuracy.
We compare the real grammar with the approximately optimized grammar that achieved the lowest AUC value in Table~\ref{tab:grammar-table-jap}; the main divergence is that desiderative suffixes are placed after the negation suffix (slot 6), whereas real Japanese orders place them before the politeness suffix (slot 5).

We conducted an error analysis comparing the real Japanese morpheme order against our approximately optimized orders.
For each grammar, we extracted the pairs of morphemes whose relative order is incorrectly predicted, excluding pairs involving low-frequency morphemes not discussed here, and including a few forms that do not agree with the dominant order described above (TODO figure out what these exceptions are).
Results are shown in Table~\ref{tab:jap-err-analysis}.
The most frequent divergence is that past and negation suffixes are consistently ordered incorrectly; this affects 88 corpus examples (out of 15K total examples).
%A few of the optimized grammars show additional divergences among high-frequency morphemes, for instance, some grammars order politeness and negation incorrectly. 

\begin{table}
    \centering
    \begin{tabular}{llllllll}
	    &	    Real & Optimized \\ \hline\hline
	    &    Stem & Stem \\ \hline
1 & suru & suru \\
2 & causative & causative \\
3 & passive/potential & passive/potential \\
4 & desiderative & desiderative \\
5 & politeness & future \\
6 & negation & politeness \\
7 & future & past \\
 & past & negation \\
 & nonfinite & nonfinite \\ 
 \hline
    \end{tabular}
    \caption{Comparing order of Japanese affixes in the observed orders (left) and according to an approximately optimized grammar (right).}
    \label{tab:grammar-table-jap}
\end{table}

\begin{table}
    \centering
    \begin{tabular}{ll|ll}
    \multicolumn{2}{c|}{Error} & Frequency \\ \hline\hline
negation & past & 88 \\
politeness & future & 9 \\
%negation & suru & 6 \\
negation & politeness & 3 \\
%politeness & passive/potential & 2 \\
%causative & suru & 2 \\
\end{tabular}
    \caption{Errors in Japanese: We show pairs of morphemes that are ordered incorrectly by the approximately optimized grammar.
    We indicate the number of such pairs occurring in the corpus.
    We only count divergences as errors here if they are predicted by the order (TODO figure out where exceptions come from). Also, we only show errors where both morphemes are among the high-frequency ones studied here.
    %\rljf{How is the frequency column calculated?}
    }
    \label{tab:jap-err-analysis}
\end{table}

\paragraph{Sesotho results.} We compare the real Sesotho grammar with the approximately optimized grammar that achieved the lowest AUC value in Table~\ref{tab:grammar-table-sesotho}.

In Sesotho, for prefixes, all optimized grammars almost exactly recover the ordering described above.
The only divergence among the high-frequency morphemes is that negation and the tense/aspect prefix are ordered incorrectly; this accounts for only 12 occurrences in the data set, as the two prefixes rarely co-occur.
Other divergences affect lower-frequency morphemes (TODO how many).
Common divergences are shown in Table~\ref{tab:sesotho-prefix-err-analysis}.

For Sesotho suffixes, order is recovered at above-chance accuracies, though with some divergences.
An error analysis is shown in Table~\ref{tab:sesotho-suffix-err-analysis}.
The most common error is that relative and interrogative suffixes are consistently placed closer to the verb stem than the mood suffix.
We conjecture that this happens because all Sesotho verbs uniformly have a mood suffix, suggesting that there might be lower mutual information between the stem and the mood suffix than between the stem and these two suffixes.
Furthermore, valence-changing suffixes are ordered farther away from the stem than various other suffixes, in contrast with the actual orders.

\begin{table}
    \centering
    \begin{tabular}{ll|ll}
    \multicolumn{2}{c|}{Error} & Frequency \\ \hline\hline
Negation & Tense/aspect & 12 \\
\\
    \multicolumn{2}{c|}{Error} & Frequency \\ \hline\hline
Mood & Interrogative & 2204 \\
Mood & Relative & 858 \\
Applicative & Tense/aspect & 347 \\
%Tense/aspect & Passive & 302 \\
Causative & Tense/aspect & 174 \\
Neuter & Tense/aspect & 155 \\
\end{tabular}
    \caption{Errors in Sesotho prefixes (top) and suffixes (bottom). We only count divergences as errors here if they are predicted by the order (TODO figure out where exceptions come from). Also, we only show errors where both morphemes are among the high-frequency ones studied here.}
    \label{tab:sesotho-prefix-err-analysis}
\end{table}




\begin{table}
    \centering
    \begin{tabular}{llllllll}
	    &	    Real & Optimized \\ \hline\hline
	    1 & Subject & Subject \\
	      & Subject (rel.) & Subject (rel.) \\
	    2 & Negation & Tense/aspect \\
	    3& Tense/aspect & Negation \\
	    4 &Object & Object \\ \hline
	    &Stem & Stem  \\ \hline
	    1 & Reversive & Passive \\
	    2& Causative & Reciprocal \\
	    &Neuter & Tense/aspect \\
	    &Applicative & Neuter \\
	    &Reciprocal & Relative \\
	    3&Passive & Causative \\
	    4&Tense/aspect & Applicative \\
	    5&Mood & Interrogative \\
	    6&Interrogative & Reversive \\
	    &Relative & Mood \\ \hline
    \end{tabular}
	\caption{Comparing order of Sesotho affixes in the observed orders (left) and according to an approximatively optimized grammar (right). Note that order was separately optimized for prefixes and suffixes.}
    \label{tab:grammar-table-sesotho}
\end{table}


%\begin{figure}
%\begin{center}
%\begin{tabular}{c||llll}
%             &       Pairs & Full \\ \hline\hline
%Optimized for Phoneme Prediction   &   0.962 (SD 0.001) & 0.957 (SD 0.002) \\
%Optimized  &   0.954 (SD 0.009) & 0.945 (SD 0.012) \\ 
%Random Baseline    &  0.504 (SD 0.0) & 0.414 (SD 0.0) \\ 
%\end{tabular}
%\end{center}
%\caption{Accuracy of approximately optimized orderings, and of random baseline orderings, in predicting verb suffix order in Japanese. `Pairs' denotes the rate of pairs of morphemes that are ordered correctly, and `Full' denotes the rate of verb forms where order is predicted entirely correctly. We show means and standard deviations over different runs of the optimization algorithm (`Optimized'), and over different random orderings (`Random').}\label{fig:acc-japanese}
%\end{figure}










\subsection{Discussion}
We have found that the ordering of verb affixes in Japanese and Sesotho provides approximately optimal memory--surprisal trade-offs.
We further found that parts of these languages' ordering rules can be derived from optimizing order for efficient tradeoffs.

We argue that the memory--surprisal trade-off provides an explanation of previously-existing typological generalizations, and an operationalization of previous functionally-motivated explanations for them.
Most prominently, \citet{bybee-morphology-1985} has claimed that a universal ordering of verbal inflectional morphemes exists across languages:
\begin{quote}
\begin{tabular}{llllllllllllllllllllllllll}
verb stem & valence & voice & aspect & tense& mood & modality & subj. person & subj.number 
\end{tabular}
\end{quote}
Morphemes are claimed either to go in the order above (suffixes), or its reverse (prefixes).

Japanese and Sesotho verb affixes are broadly in agreement with Bybee's generalization.
For instance, valence and voice suffixes are closer to the stem than tense/aspect/mood markers.
Subject agreement in Sesotho is farther away from the verb than tense/aspect/mood prefixes.
This ordering is reproduced closely by optimization in Japanese and for Sesotho prefixes, and to some extent also for Sesotho suffixes.

Bybee (p. 37) argues further that morpheme order is determined by the degree of \emph{relevance} between the affix and the stem, that is, the degree to which ``the semantic content of the first [element] directly affects or modifies the semantic content of the second'' (p. 13).
She argues that elements whose meanings are more relevant to each other appear closer together.
For instance, the meaning of a verb is impacted more strongly by a causative affix than by a tense affix:
Combining a verb with a causative marker results in a form that denotes a different action, whereas a tense affix only locates the action in time.

We conjecture that this notion of relevance is related to mutual information.
If an affix has a stronger impact on the meaning of the verb, it will typically not be applicable to all verbs.
For instance, causative markers will only attach to verbs whose semantics is compatible with causation.
In contrast, a past tense marker can attach to all verbs that are compatible with actions that can have occurred in the past.
Therefore, we expect that affixes that are more relevant to a verb stem will also tend to have higher mutual information with the verb stem.
If they have higher mutual information with the verb stem, then the principle of information locality predicts that they will go close to the verb stem.



\section{General Discussion}\label{sec:discussion}

\paragraph{The Roles of Speakers, Listeners, and Interaction}
Our derivation of the memory-surprisal tradeoff considers the listener.

Debate about the roles of speaker and listener in shaping language.

Interaction: We could make the process include the entire conversational history (including what the listener said before).


\paragraph{`hockeystick problem'} conceptual issue: we're agnostic as to whether the actual full surprisal is different between real and baseline (we're restricting context length to $T$, and networks might not extract all the available information). all we're saying is that, in the setting of small memory budgets, real languages provide lower surprisals.

logically different possibilities

- same asymptotic surprisal, different tradeoff

- same small-memory tradeoff, different asymptotic surprisal (difference might come out later)

- better small-memory tradeoff but worse high-memory tradeoff

\paragraph{Nature of the Bound}
We have a lower bound, not an exact estimate of the tradeoff curve

\paragraph{Extralinguistic Context}
The assumption about information flow disregards the role of information sources that are external to the linguistic material in the sentence.
For instance, the interlocutors might have common knowledge of the weather, and the listener might use this to construct predictions for the speaker's utterances, even if no relevant information has been mentioned in the prior discourse.
Such sources of information are disregarded in our model.

\paragraph{Capacity vs Retrieval}
Our theoretical analysis places the main memory bottleneck in the capacity of short-term memory.
Not all models of memory in sentence processing make this assumption.
Indeed, there is evidence that difficulty of retrieving items is an important bottleneck in sentence processing.
In SI Section X, we show that our theoretical bounds are also compatible with a retrieval-based model such as ACT-R.


\paragraph{Limitations of Grammar Model}

\paragraph{absolute numbers, how they relate to what's known about human memory}

\subsection{Relation to Models of Sentence Processing}
\label{sec:sentprod-models}

There is a substantial literature proposing sentence processing models and quantitative memory metrics for sentence processing.
In this section, we discuss how our theoretical results relate to and generalize these previously proposed models.
We do not view our model as competing with or replacing any of these models; instead, our information-theoretic analysis captures aspects that are common to most of these models and shows how they arise from very general modeling assumptions. 
%In this section, we argue that---although our theory is couched in the language of surprisal theory \citep{hale2001probabilistic,levy2008expectation,hale2016information}---the memory--surprisal curve reflects fundamental information-theoretic trade-offs that must apply in any theory. 

In Section~\ref{sec:tradeoff}, we proved a formal relationship between the entropy of memory $H_M$ and average surprisal $S_M$. 
We have made no assumptions about the architecture of incremental memory, and so our result is general across all such architectures.
Memory representations do not have to be rational or optimal for our bound in Theorem~\ref{prop:suboptimal} to hold.
There is no physically realizable memory architecture that can violate this bound.


However, psycholinguistic theories may differ on whether the entropy of memory $H_M$ really is the right measure of memory load, and on whether average surprisal $S_M$ really is the right predictor of processing difficulty for humans. Therefore, in order to establish that our information-theoretic processing model generalizes previous theories, we will establish two links:
\begin{itemize}
    \item Our measure of memory usage generalizes theories that are based on counting numbers of objects stored in incremental memory \citep[e.g.,][]{yngve1960model,miller-finitary-1963,frazier1985syntactic,gibson-linguistic-1998,kobele2013memory,graf2014evaluating,GrafEtAl15MOL,gerth2015memory,GrafEtAl17JLM,desanto2020parsing}. Furthermore, for theories where memory is constrained in its capacity for \emph{retrieval} rather than storage \citep[e.g.,][]{mcelree-memory-2003,lewis-activation-based-2005}, the information locality bound will still hold.
    \item Our measure of processing difficulty (i.e., average surprisal) reflects at least a \emph{component} of the predicted processing difficulty under other theories.
\end{itemize}

More specifically, our measure of memory usage is based on mutual information, and so it is equivalent to counting the number of objects stored in memory, weighted by their amount of predictive information.

In this section, we will argue that any realistic theory of sentence processing must include surprisal as at least a \emph{component} of the cost of processing a word, even if it is not explicitly stated as such. 
%More formally, we argue that processing difficulty for a word $w_t$ must at least be given by $k (-\log P(w_t |m_t)) + R$, with $k$ a scaling factor giving the relative contribution of surprisal to processing cost, and $R$ representing all other factors. The scaling factor $k$ may be small, but it cannot be zero, for both empirical and theoretical reasons. 
We make this argument on both empirical and theoretical grounds.
Empirically, surprisal makes a well-documented and robust contribution to processing difficulty in empirical studies of reading times and event-related potentials \citep{smith2013effect,frank2016erp}. 
Theoretically, surprisal may represent an irreducible thermodynamic cost incurred by any information processing system \citep{landauer,still2012thermodynamic,zenon2019information}, and there are multiple converging theoretical arguments for why it should hold as a cost in human language processing in particular \citep{levy2013memory}. 




Here, we discuss the concrete connections between our sentence processing model and existing models of human sentence processing. 




\paragraph{The Dependency Locality Theory}
We studied average surprisal in systems with bounds on memory usage. 
Our lower bound on memory usage, described in Theorem~\ref{prop:suboptimal} Eq.~\ref{eq:memory-bound}, is formally similar to Storage Cost in the Dependency Locality Theory (DLT) \citep{gibson-linguistic-1998,gibson2000dependency}.
In that theory, storage cost at a given timestep is defined as the `number of predictions' that are held in memory.
Our bound on memory usage is stated in terms of mutual information, which indicates the amount of predictive information extracted from the previous context and stored in memory.
Therefore, our measure generalizes DLT storage cost. 
Our meaasure goes beyond DLT storage cost in that the DLT quantity only considers predictions that are certain, and each prediction takes an equal amount of memory; in contrast, our measure of memory usage can be seen as weighting predictions by their certainty and amount of predictive information. 
In this sense, DLT storage cost can be seen as an approximation to Eq.~\ref{eq:memory-bound}.

The other component of the DLT is integration cost, the amount of difficulty incurred by establishing a long-term syntactic dependency. In our framework, DLT integration cost corresponds to surprisal given an imperfect memory representation, following \cite{futrell2020lossy}.

There is one remaining missing link between our theory of processing difficulty and theories such as the Dependency Locality Theory:
our information locality theorem says that \emph{statistical} dependencies should be short-term, whereas psycholinguistic theories of locality have typically focused on the time-span of \key{syntactic dependencies}: words which depend on each other to determine the meaning or the well-formedness of a sentence. Statistical dependencies, in contrast, mean that whenever one element of a sequence determines or predicts another element \emph{in any way}, those two elements should be close to each other in time. 

If statistical dependencies, as measured using mutual information, can be identified with syntactic dependencies, then that would mean that information locality is straightforwardly a generalization of dependency locality. \citet{futrell2019syntactic} give theoretical and empirical arguments that this is so. They show that syntactic dependencies as annotated in dependency treebanks identify word pairs with especially high mutual information, and give a derivation showing that this is to be expected according to a formalization of the postulates of dependency grammar. The connection between mutual information and syntactic dependency has also been explored by \citet{}. % maximizing mutual information principle; de paiva alves; yuret


\paragraph{Cue-Based Retrieval Models}

In some psycholinguistic theories, memory-related difficulty arises not because of a bound on memory capacity, but rather because of difficulties involved in retrieving information from memory \citep{}. We are able to prove an analogous theorem that applies to such theories: see General Discussion for an extension of our analysis to the case involving a short-term memory (STM) with unbounded capacity, a working memory (WM) with limited capacity, and cost associated with communication between WM and STM. Essentially, the constraint on the the memory state in our theorem above can be re-interpreted as a constraint on the capacity of a communication channel linking STM to WM. In particular, this result constrains average surprisal for memory models based on cue-based retrieval such as the ACT-R model of \citet{lewis-activation-based-2005}.



The ACT-R model of \cite{lewis-activation-based-2005} does not have an explicit surprisal cost.
Instead, surprisal effects are interpreted as arising because, in less constraining contexts, the parser is more likely to make decisions that then turn out to be incorrect, leading to additional correcting steps.
%We see this as an algorithmic-level implementation of the justification for surprisal theory provided by \citet{levy2008expectation}:
We view this as an algorithmic-level implementation of a surprisal cost $H[x_t|m_{t-1}]$:
If the word $x_t$ is unexpected given the current state of the working memory -- i.e., buffers and control states -- then their current state must provide insufficient information to constrain the actual syntactic state of the sentence, meaning that the parsing steps made to integrate $x_t$ are likely to include more backtracking and correction steps.
Thus, we argue that cue-based retrieval models predict that the surprisal $- \log P(x_t|m_{t-1})$ will be part of the cost of processing word $x_t$.



See General Discussion for a more detailed discussion how how average surprisal can describe processing cost in ACT-R models of sentence processing.




- small buffer, unbounded unstructured store of chunks

- content-based retrieval

- interference


In cue-based retrieval models, memory effects arise from retrieval interference, whereas surprisal effects arise from backtracking (CITE).
Our model implements a bound on capacity, which is usually understood to be different from the presence of retrieval interference.
See SI Section X for a version of our model that more closely parallels cue-based retrieval models.

could also have more extensive discussion of the links here instead of the SI


\paragraph{Lossy-Context Surprisal}
\citet{futrell2020lossy} describe a processing model where listeners make predictions (and incur surprisal) based on lossy memory representations.
In particular, they consider loss models that delete, erase, or replace words in the past.
Within this model, they were able to establish a similar information locality result, by showing that the theoretical processing difficulty increases when words with high \emph{pointwise mutual information} are separated by large distances \citep{futrell-noisy-context-2017,futrell2019information}. Pointwise mutual information is the extent to which a \emph{particular value} predicts another value in a joint probability distribution. For example, if we have words $w_1$ and $w_2$ in a sentence, their pointwise mutual information is:
\begin{equation*}
    \text{pmi}(w_1; w_2) \equiv \log \frac{P(w_2|w_1)}{P(w_2)}.
\end{equation*}
Mutual information, as we defined it in Eq.~\ref{eq:mi}, is the \emph{average} pointwise mutual information over an entire probability distribution.

Our information locality bound theorem differs from this previous result in three ways:
\begin{enumerate}
    \item \citet{futrell2020lossy} required an assumption that incremental memory is subject to decay over time. In contrast, we do not require any assumptions about incremental memory except that it has bounded capacity (or that retrieval operations have bounded capacity; see below).
    \item Our result is a precise bound, whereas the previous result was an approximation based on neglecting higher-order interactions among words.
    \item Our result is about the fall-off of the mutual information between words, conditional on the intervening words. The previous result was about the fall-off of \emph{pointwise} mutual information between specific words, without conditioning on the intervening words.
\end{enumerate}

Many proposals for memory metrics are based on the \emph{number} of objects simultaneously held in a stack-like data structure of memory, and \emph{how long} they are stored.

Our theoretical result can be seen as an information-theoretic version of such metrics.
It differs from these in two ways:
First, it considers not only the cost of maintaining direct syntactic dependencies, but extends this to any statistical relations between words that can be exploited for prediction.
Second, it weights these relations by the amount of predictive information shared between the words.

head-dependent mutual information hypothesis
In an idealized model \emph{all} predictive information is mediated through direct syntactic dependencies, and each dependency contributed an equal amount of predictive information, our memory bound would indeed become equivalent to such a metric.
Note that the average number of objects held in memory simultaneously is equivalent to the average time they are stored.



\cite{yngve1960model} production model with memory complexity measure (but problematically predicts left-branching structures to be hard, Kimball (1973))

\cite{miller-finitary-1963}: degree of self-embedding limited

\cite{frazier1985syntactic} local nonterminal count

Kimball (1973): Principle of Two Sentences

Rambow and Joshi 1994 using TAG

Marcus 1980 deterministic parsing

(Sabrina Gerth, Memory Limitations in Sentence Processing)

\cite{gerth2009unifying}


\cite{wanner1978atn}
\cite{frazier1978sausage}

\cite{rambow201512}: count words on stack

\cite{boston2012computational}

\cite{just1992capacity}

\cite{marcus1978theory}: deterministic parser with small lookahead






\paragraph{Metrics based on Minimalist Parser Models}
memory metrics based on minimalist parsers

\cite{kobele2013memory}
\cite{graf2014evaluating}
\cite{GrafEtAl15MOL}
\cite{gerth2015memory}
\cite{GrafEtAl17JLM}
\cite{desanto2020parsing}
based on counting how long a node is kept in memory and how many nodes are kept in memory

Linear or ranked combinations of versions of these metrics have been argued to account for various complexity and acceptability differences in complex embeddings.

\paragraph{Models unifying memory and surprisal}


evaluating memory metrics and surprisal on reading measures \cite{boston2008parsing} \cite{demberg2008data} \cite{boston2011parallel}

- Demberg et al: unified parsing model with both integration and surprisal costs \cite{demberg2009computational,demberg2013incremental}

- lossy-context surprisal: the perspective taken here

\paragraph{Entropy Reduction}


\subsection{Statistical Studies of Language}

\paragraph{Statistical Complexity}
Our formalization of listener memory is related to studies of dynamic systems in the Physics literature.
The tradeoff between listener memory and surprisal is formally equivalent to the \emph{Recursive Information Bottleneck} considered by \cite{still-information-2014}.
In the limit of optimal prediction and minimal surprisal, our formalization of listener memory is equivalent to the notion of \emph{Statistical Complexity} \citep{crutchfield-inferring-1989}.
In the limit $T \rightarrow \infty$, the quantity in (\ref{eq:memory}) is equal to the \emph{excess entropy} \citep{crutchfield-inferring-1989}.
However, the link between memory and information locality provided by our Theorem~\ref{prop:suboptimal} appears to be a novel theoretical contribution.
Relatedly, \cite{sharan-prediction-2016} shows a link between excess entropy and approximability by $n$-th order Markov models, noting that processes with low excess entropy can be approximated well with Markov models of low order.


\paragraph{Decay of Mutual Information}
In Propositions~\ref{prop:lower-bound} and \ref{prop:suboptimal}, we showed a close link between memory and the decay of \emph{conditional} mutual information $I_t := I[w_t, w_0 | w_{1\dots t-1}]$.
Prior work has studied the decay of \emph{unconditional} mutual information $I[w_t, w_0]$ in natural language \citep{ebeling-entropy-1994,lin-critical-2017}, and linked it to locality and memory \citep{futrell-noisy-context-2017}.

The decay of unconditional mutual information is less closely linked to memory requirements than conditional mutual information:
While the decay of conditional mutual informations provides a lower bound on memory need, unconditional mutual information does not:
Consider the constant process where with probability 1/2 all $w_t = 0$, and with probability 1/2 all $w_t = 1$. %%$w_t = c$, where $c$ is random but independent of $t$ for each specific draw from the process.
The unconditional mutual information is 1 at all distances, so does not decay at all, but the process only requires 1 bit of memory.
Conversely, one can construct processes where the unconditional mutual informations are 0 for all $t$, but where $P > 0$ and this predictive information is actually spread out over arbitrarily large distances (that is, the ratio of memory $M$ and predictability $P$ can be made arbitrarrily large).\footnote{First, consider the process (called X by REF) consisting of 2 random bits and their XOR. This one has bounded nonzero $J$, but zero unconditional MI. To get unbounded $J$, consider the following process for any $N \in \mathbb{N}_{>2}$: Every $w_t$ is equal to the XOR of $w_{t-1}$ and $w_{t-N}$, such that each $w_t$ has $Bernoulli(1/2)$ as its marginal. The unconditional mutual information between any two timesteps is zero, but modeling the process requires $N$ bits of memory.}



\paragraph{Long-range dependencies in text}    % excess entropy
\cite{debowski-excess-2011} has studied the excess entropy of language across long ranges of text, in particular studying whether it is finite. % compute excess entropy in text
Our work contrasts with this work in that we are interested in dependencies within sentences.


\paragraph{Decay vs Interference}
Work has suggested that interference and memory overload is more appropriate than decay \cite[p. 408]{lewis-activation-based-2005} for modeling locality and memory in sentence processing.
The bounds in Propositions~\ref{prop:lower-bound} and \ref{prop:suboptimal} hold for any type of memory model, and are thus compatible with decay- or interference-based models.
The formula in (\ref{eq:memory-bound}) might suggest that boundedness of memory entails that memory has to decay.
This is not the case:
A long dependency can be maintained perfectly with low average memory:
Informally, if every sentence is $N$ words long and has one long-distance dependency spanning the entire sentence, this dependency can be modeled perfectly with a memory cost that is independent of $N$.
In contrast, if every symbol strongly and non-redundantly depends on the character $T$ steps in the past, with $T$ large, this will create a memory cost proportional to $T$.




\paragraph{Memory and Hierarchical Structure; Finiteness of Memory}
Processing nontrivial hierarchical structures typically requires unbounded amounts of memory.
However, crucially, the \emph{average} memory demand for prediction can be finite, if the probability mass assigned to long dependencies is small.
For instance, languages defined by Probabilistic Context Free Grammars (PCFG) always have finite average memory.
The reason is that PCFGs assign low probabilities to long sequences.\footnote{Proposition 2 in \cite{chi-statistical-1999} implies that words drawn from a PCFG have finite expected length. This implies that average memory demands are finite.}



%\paragraph{Center Embeddings}
%\cite{miller-finitary-1963} attributed the unacceptability of multiple center-embedding to memory limitations.
%\cite{gibson-linguistic-1998}
%\paragraph{Other Psycholinguistic Predictions}
% RF: the fact that you would get locality effects given medium WM capacity, but not very high or very low WM capacity, as Bruno Nicenboim found. And maybe some speaker-listener asymmetries. 
%\paragraph{Speakers}
% RF: what matters for the speaker is not I[w_t, w_0 | w_1, …, w_{t-1}], but I[w_t, w_0 | w_1, …, w_{t-1}, G] where G is some representation of the speaker’s goal (like in the van Dijk paper). This changes the interpretation of the mutual information. For the listener, it’s just redundancy. For the speaker, it’s redundancy *conditional on the goal*—which you could interpret as something like conceptual relatedness of linguistic elements. Then the speaker’s pressure is to keep conceptually related things close. 





\subsection{Remarks}


\paragraph{Are distant words forgotten?} Our theorem shows that a listener will inevitably be affected by surprisal cost corresponding to dependencies longer than some timescale $T$. However, it does not necessarily imply that a listener forgets words beyond some amount of time $T$ in the past. An optimal listener may well decide to remember information about words more distant than $T$, but in order to stay within the bounds of memory, she can only do so at the cost of forgetting some information about words closer than $T$.
The Information Locality Lower Bound still holds, in the sense that the long-term dependency will cause processing difficulty, even if the long-term dependency is not itself forgotten.
See SI Appendix Section X for a mathematical example illustrating this phenomenon.

\subsection{Memory--surprisal trade-off in language production}

An analogous memory--surprisal trade-off exists in language production. In this case, the trade-off arises from the minimization of error in production of linguistic sequences. That is, given a \key{competence language} (a target distribution on words given contexts), a speaker tries to produce a \key{performance language} which is as close as possible to the competence language. The performance language operates under memory constraints, so the performance language will diverge from the competence language due to production errors. When a speaker has more incremental memory about what she has already produced, then she is able to produce linguistic sequences with less error, thus reducing the divergence between the performance language and the competence language. The reduction of this competence--performance divergence is formally equivalent to the minimization of average surprisal from Section~\ref{sec:listener-tradeoff}.

% TODO maybe cite some Karl Lashley stuff? Maryellen MacDonald? Shota Momma?

We derive the existence of this trade-off from the following postulates about language production. Let the competence language be represented by a stationary stochastic process, parameterized by a probability distribution $p(w_t | w_{<t})$ giving the conditional probability of any word $w_t$ given an unbounded number of previous words. Our postulates describe a speaker who tries to find a performance language $q(w_t|m_t)$ to match the the competence language using incremental memory representations $m_t$:

\begin{enumerate}
    \item Production Postulate 1 (Incremental memory). At time $t$, the speaker has an incremental \key{memory state} $m_t$ that contains (1) her stored information about previous words that she has produced, and (2) information about her production target. The memory state is given by a \key{memory encoding function} $M$ such that $m_t = M(w_{t-1}, m_{t-1})$.
    
    \item Production Postulate 2 (Production policy). At time $t$, the speaker produces the next word $w_t$ conditional on her memory state by drawing from a probability distribution $q(w_t | m_t)$. We call $q$ the speaker's \key{production policy}.
    
    \item Production Postulate 3 (Minimizing divergence). The production policy $q$ is selected to minimize the KL divergence from the performance language to the target competence language $p(w_t|w_{<t})$. We call this divergence the \key{competence--performance divergence} under the memory encoding function $M$ and the production policy $q$:
    \begin{align}
    \label{eq:comp-perf-div}
    d^q_M &\equiv D_{\text{KL}} [ p(w_t|w_{<t}) || q(w_t|m_t) ] \\
        &= \sum_{w_{\le t}} p(w_{\le t}) \log \frac{p(w_t | w_{<t})}{q(w_t|m_t)}.
    \end{align}
    The production policy is then the solution to the functional minimization problem:
    \begin{equation}
        \mathop{\text{minimize }}_{q(w_t|m_t)} d^q_M.
    \end{equation}
\end{enumerate}

Completing the link with the memory--surprisal trade-off in comprehension, we note that when the production policy $q(w_t|m_t)$ is selected to minimize the competence--performance divergence $d^q_M$, then this divergence becomes equal to the memory distortion $d_M$ discussed above in the context of comprehension costs. Therefore, under these postulates, the Information Locality Bound Theorem will apply in production as well as comprehension. This means that languages that exhibit information locality can be produced with greater accuracy given limited memory resources.

Although the memory--surprisal trade-off is mathematically equivalent between comprehension and production, its psycholinguistic interpretation is different. In the case of language comprehension, the trade-off represents excess processing \emph{difficulty} arising due to memory constraints. In the case of language production, the trade-off represents \emph{production error} arising due to memory constraints. When memory is constrained, then the speaker's productions will diverge from her target language. And as memory is more and more constrained, this divergence will increase more and more. The degree of divergence is measured in the same units as surprisal, hence the formal equivalence between the listener's and speaker's memory--surprisal trade-offs. 


\section{Conclusion}\label{sec:conclusion}

In this work, we have provided evidence that human languages order elements in a way that reduces cognitive resource requirements, in particular memory effort.
We provided an information-theoretic formalization of memory requirements as a tradeoff of memory and surprisal.
We showed theoretically that languages have more efficient tradeoffs when they show stronger degrees of information locality.
Information locality provides a formalization of various locality principles from the linguistic literature, including dependency locality \citep{gibson1998linguistic}, domain minimization \citep{hawkins2004efficiency}, and the proximity principle \citep{givon1985iconicity}.
Using this result, we provided evidence that languages order words and morphemes in such a way as to provide efficient memory--surprisal tradeoffs.
Therefore, the memory--surprisal tradeoff simultaneously provides (1) a unified explanation of diverse typological phenomena which is rigorously grounded in the psycholinguistics literature, (2) a theory which makes new successful quantitative predictions about word and morpheme order within and across languages, and (3) a mathematical framework relating universals of language to principles of efficient coding from information theory. 

Our result shows that wide-ranging principles of order in natural language can be explained from highly generic cognitively-motivated information-theoretic principles. The locality properties we have discussed are some of the most characteristic properties of natural language, setting natural language apart from other codes studied in information theory.
Therefore, our result raises the question of whether other distinctive characteristics of language---for example, mildly context-sensitive syntax, duality of patterning, and compositionality---might also be explained in terms of information-theoretic resource constraints.



\bibliographystyle{apalike}
\bibliography{literature}

\end{document}






