A wide range of work has argued that information in natural language utterances is ordered in ways that reduce memory effort.
This strand of research has brought together linguistic typologists documenting language universals, corpus linguists studying production preferences, psycholinguistics studying processing difficulty, and computer scientists studying formal properties of grammars.

An early example is \cite{miller-finitary-1963}, who attributed the unacceptability of multiple center embeddings in English to limitations of human working memory; following work has studied how different grammars induce different memory requirements in terms of the number of symbols that must be stored at each point to produce or parse a sentence \citet{yngve1960model,yngve1961depth,abney1991memory,gibson1991computational,resnik1992left}.

For example, \citet{grodner-consequences-2005} studied word-by-word reading times in a series of sentences such as~\ref{ex:grodner} below. 
\eenumsentence{\label{ex:grodner}
\item The \underline{administrator} who the nurse \underline{supervised}\dots\label{ex:grodner1}
\item The \underline{administrator} who the nurse from the clinic \underline{supervised}\dots
\item The \underline{administrator} who the nurse who was from the clinic \underline{supervised}\dots\label{ex:grodner3}
}
In these sentences, the distance between the noun \emph{administrator} and the verb \emph{supervised} is successively increased. \citet{grodner-consequences-2005} found that as this distance increases, there is a concomitant increase in reading time at the verb \emph{supervised} and following words. The hypothesized reason is the following: at the word \emph{supervised}, a comprehender who is trying to compute the meaning of the sentence must integrate a representation of the verb \emph{supervised} with a representation of the noun \emph{administrator}, which is a direct object of the verb. This integration requires retrieving the representation of \emph{administrator} from working memory. If this representation has been in working memory for a long time---for example as in Sentence~\ref{ex:grodner3} as opposed to \ref{ex:grodner1}---then the retrieval operation causes difficulty that manifests as increased reading time. There exists a dependency between the words \emph{administrator} and \emph{supervised}, and excess processing difficulty occurs when these two words are separated. 


\cite{hawkins-efficiency-2003} provides cross-linguistic evidence that word orders are optimized for processing based on local contexts.
Further work has found computational, corpus-based evidence that memory limitations impact language structure and production.
In particular, languages have been shown to shorten the length of syntactic dependency lengths \citep{futrell-large-scale-2015}.
Dependency length can be linked to memory use in certain models of incremental syntactic parsing, and increases processing difficulty in theories of memory in sentence processing \citep{gibson-linguistic-1998}.
\cite{gildea-human-2015} further provide evidence from five languages that word orders optimize predictability from local contexts.
\cite{futrell-noisy-context-2017} provide evidence that language shows information locality, i.e., elements with higher mutual infomation are closer together, which is predicted by their model of Lossy-Context Surprisal.

All these models of memory in sentence processing, and derived measures of efficiency for memory allocation, require specific assumptions about the architecture of memory.
This leaves open the question whether such assumptions are necessary, or whether word orders across languages are optimized for memory independently of the implementation and architecture of human language processing.
For example, models of memory in sentence processing differ in whether they assume limitations in capacity \citep{gibson1998linguistic}, the precision at which information is encoded (CITE), or the precision with which specific elements can be retrieved from memory (CITE). From an information-theoretic perspective, these options are different instantiations of the same quantity, namely, the mutual information between past input and the representations that are used for processing new material.
\mhahn{maybe it would be beneficial to, at some point (Discussion), illustrate how limitations in our framework are instantiated by memory limitations in different frameworks, such as DLT, ACT-R, Marking\&Morphing, ...}

We approach this question by first providing general information-theoretic lower bounds on memory load that will hold independently of the architecture of memory representations.
We will consider a general setting of a listener performing incremental prediction.
Our result immediatly entails a link between boundedness of memory and locality, which had been stipulated or derived from assumptions about memory architecture in previous models \citep{gibson-linguistic-1998, lewis-activation-based-2005, futrell-noisy-context-2017}.
We will then use corpus data from over 52 languages to provide evidence that their word orders help lower memory cost.


% Production argument
% Idea: A producer wants to approximate a language conditional on a production target G.
% Ie producer finds m_t to minimize loss: D[ w_t | w_{<t}, g   ||   w_t  | m_t  ] + a * H[m_t]
% That is, the memory has to contain both the previous words and the current production goal.
% The loss comes out to I[ w_t : w_{<t},g | m_t] + a * H[m_t]
%                       = I[w_t : w_{<t} | m_t] + I[w_t : g | w_{<t}, m_t] + a * H[m_t]
% Now let's decompose the memory m_t into two parts:
% the part about the previous words: call this r_t
% the part about the current goal: call this g_t. And let's assume H[g|g_t]=0, i.e. the memory stores
% all the information about the goal.
% Then H[m_t] = H[r_t] + H[g_t|r_t].
% Now we have loss:
%     I[w_t : w_{<t} | r_t, g] + I[w_t : g | w_{<t}, r_t, g] + a H[r_t, g_t]
%   = I[w_t : w_{<t} | r_t, g] + a H[r_t, g_t] 
% Now we have H[r_t, g_t] >= H[r_t] >= \sum t I_t.
% And then we have 

%TODO mention prior work
%
%-  center embeddings dispreferred/impossible to process \cite{miller-finitary-1963}
%
%- explanation of greenberg universals
%
%- dependency length minimization
%
%- \cite{gibson-linguistic-1998}
%

%\paragraph{\cite{berwick-grammatical-1986}}
%They considered incremental parsing, or, more precisely, incremental recognition of grammaticality.



%also (CITE): operations in a specific incremental parsing model


