- datasets

-- unimorph

-- bibles corpus?

-- celex


- permute phonemes

- permute syllables

- permute phonemes only within syllables

- permute morphemes 

\section{Morphology}

Bybee 1985:

verb-valence-voice-aspect-tense-mood-modality-subj.person-subj.number

can also directly quantify MI with verb stem (presence \& choice for each slot) as a sanity check


\subsection{Japanese}


\paragraph{Data Selection}

UD corpora 2.4 with freely available word forms. Only training set, so only GSD.

Verb forms are tokenized so that most suffixes are segmented as individual words.

We selected all chains VERB AUX...AUX. If a -te followed, added this.

We obtained 15281 forms, with a median number of X suffixes (max=...).

Describe format: form vs lemma.


Japanese verb stems and suffixes show alternations conditioned on neighboring suffixes.
In the GSD treebank, each stem and suffix is annotated with a `lemma' indicating a normalized context-independent representation of the morpheme.

In the UD treebanks, each morpheme is 


\paragraph{Verb Suffixes in Japanese}

Japanese orthography does not indicate word boundaries, and there is no universally accepted segmentation.

We determined a set of common morphemes as follows.
We selected all morphemes occurring at least 50 times and annotated their meaning/function.
Also, we excluded three morphemes that are treated as independent words, not suffixes, by \cite{kaiser2013japanese} (dekiru, naru, yoo).
Also, passive and potential markers are formally identical for many verbs; we included both here.

Also, we took suru to be a unit with the preceding element (it is used for turning Sino-Japanese words into verbs).

We list the morphemes according to the order extracted according to the model.
While this ordering matches almost all observed forms, there are some orderings that do not fit into this pattern, see below.



TODO read Kaiser p. 398 potential

\begin{enumerate}
\item VALENCE: causative (-ase-). \cite[142]{hasegawa2014japanese} \cite[Chapter 13]{kaiser2013japanese} (saseru, seru. 190)
\item VOICE: passive (-areru, -rareru). passive \cite[152]{hasegawa2014japanese} \cite[Chapter 12]{kaiser2013japanese} -areru, -rareru. (rareru, reru. 2000)
\item POTENTIAL: (e.g., -eru in UD)  atsuka-e-nai `can't handle' %扱えない GSD treebank
araso-e-nai `can't dispute' % 争えない GSD treebank

how is this ordered relative to PASSIVE? In UD, they don't seem to be always separated (ambiguity of reru/rareru. \cite[346]{vaccari1938complete}: identical to the passive for one class, or -eru for other verb class).
\item POLITENESS (masu, mashi, mase. 600) \cite[190]{kaiser2013japanese}. % 見できます, 見ましたい

It is interesting that this is relatively close to the verb

\item MODALITY: -ta- (-tai, -taku-, -taka-) desiderative (tai. 85) \cite[238]{kaiser2013japanese}. mi-mashi-tai `I want to see'
\item NEGATION: negation (-nai, -n-, -nakaC-. 630). also -mai for negative+volition.
\item TENSE/ASPECT/MOOD:

-ta for past (4000)

-yoo for hortative, future, ... 92 \cite[229]{kaiser2013japanese}. Q how do-nai- and -yoo interact?
\item -te derives a nonfinite form \cite[186]{kaiser2013japanese}. 4000
\end{enumerate}

Parallel to Bybee's hierarchy: valence closest, then voice.

Difference from Bybee' hierarchy: mood/modality are ordered closer than tense/aspect.

EXAMPLES:

\begin{tabular}{lllllllll|lllllll}
     & VALENCE   & VOICE    & \multicolumn{3}{c}{MOOD/MODALITY} & NEGATION & TENSE \\
Stem & Causative & Passive & Potential & Politeness & Desiderative & Negation & TAM & -te & \\ \hline
mi &          &      & &            &          & naka     & tta &    & did not see \cite[153]{vaccari1938complete}\\
mi &          &      & &            & taku     & nai      &     &     & I do not wish to see \cite[98]{vaccari1938complete} \\
mi &          &      & &            & taku     & naka    & tta  &      & I did not wish to see \cite[98]{vaccari1938complete} \\
oyog & ase    &      & &            &          &         & ta    &     & made swim \\
tat & ase     & rare & &            &          &         & ta    &     & was made to stand up \cite[396]{kaiser2013japanese} \\
waraw  &       & are & &            &           &         & ta    &    & was laughed at \cite[384]{kaiser2013japanese} \\
mi     &       & rare& & mase       &          & n        &      &     & is not seen \cite[337]{vaccari1938complete} \\
mi     &       & rare& & mash      &           &          & yoo  &    & will be seen \cite[337]{vaccari1938complete} \\
de     &       &    & &            &           & naka     & roo  &    & will not go out \cite[170]{vaccari1938complete} \\
mi     &       &    & e & mase       &           & n        &      &    & cannot see \cite[349]{vaccari1938complete} \\
mi     &       &    & & mashi      & tai       &          &      &    & want to see
\end{tabular}


Not all combinations are possible. 
`was not seen' is miraremasen deshita \cite[337]{vaccari1938complete}, not MIRU-RERU-MASE-NAI-TA
Negation is not directly combined with -yoo.

Some relatively frequent morphemes that are below the threshold

\begin{enumerate}
\item nakereba `if not' (cf \cite[9.3.1.4]{kaiser2013japanese})
\item tsudzukeru (continuative, 30)
\item yasui `easy to'
\item hajimeru (inchoative, 21)
\item naru (`become', 102). Maybe this should be excluded, not treated as an affix in the books.
\item yooda (180) EVIDENTIAL? yooda. but past -ta- can be after it: iruyoodatta IRU-YOODA-TA (Kaiser et al 9.5.6.1.1.1) vs misenakattayooda MISERU-NAI-TA-YOODA `it didn't seem' (GSD treebank). Maybe this should be thought of better as an SCONJ at least when including -ni, Vaccari p. 238-241.
\item beki Necessitative \cite[248]{kaiser2013japanese}
\item soo `likely to' \cite[258]{kaiser2013japanese}, probably not AUX in UD
\item dekiru potential (180), 
\end{enumerate}


there's just too many verb forms, here limited to frequent ones we found in the corpus.


\paragraph{Experiment}

modes:

- predicting on the phoneme level

- predicting on the grapheme level: equivalent to phonemic, but in units of syllables

- predicting on the level of morphemes: we take the lemmas, and unite saseru+seru to CAUSATIVE and reru+rareru+eru+keru to PASSIVE/POTENTIAL.

This controls for allomorphy / morphophonemic changes, which are conditioned on surrounding affixes (so naturally favor locality).


comparison to baselines

- random grammars

optimization:

- phonemes

- graphemes

- morphemes



% volitional 加わろう kawawar-oo
% volitional polite 加わりましょう kawawar-i-mashy-oo
% volitional neg 加わるまい(+), 加わらない[よう/こと]にしよう kawawaru-mai, kawawara-nai-yoo-nishyoo
% volitional neg polite 加わりますまい, 加わらない[よう/こと]にしましょう  kawawari-masu-mai, kawawar-anai-yoo-nishimashyoo

%  verb-valence-voice-aspect-tense-mood-modality-person-number

% http://nihongo.monash.edu/cgi-bin/wwwjdic?1W%B2%F1%B5%C4%A4%CB%B2%C3%A4%EF%A4%EB_v5r
% passive negative
% 加わられない kuwawar-are-nai

% passive negative polite
% 加わられません kuwawar-are-mase-n

% causative polite
% 加わらせます kuwawar-ase-masu

% causative negative polite
% 加わらせません kuwawar-ase-mase-n
% 加わらしません kuwawar-ashi-mase-n

% da/de/na (lemmatized as da): copula (?). na after adjectives.




honorific -masu

-s-are-naka-tta ?passive+negative.past?



% -u/-ru (non-past) -- not segmented off in UD?

% suru - reru - ta
% suru -masu - ta
% reru - ta
% suru - ta
% masu - ta
% rareru - ta
% masu - nai - deshita
% suru - reru
% suru - nai
% masu - nai
% reru - masu - ta
% dekiru - masu

% TODO in excluding -te-, have to deal with -de- as in 読んで yonde (Vaccari p. 109)

%(('し/動詞/し', 'た/助動詞/た'), 14389)
%(('る/語尾/る',), 13880)
%(('あ/動詞/あ', 'る/語尾/る'), 11481)
%(('っ/語尾/っ', 'た/助動詞/た'), 9378)
%(('さ/動詞/さ', 'れ/助動詞/れ', 'た/助動詞/た'), 7143)
%(('さ/動詞/さ', 'れ/助動詞/れ', 'る/語尾/る'), 1966)
%(('られ/助動詞/られ', 'る/語尾/る'), 1156)
%(('だっ/助動詞/だっ', 'た/助動詞/た'), 1151)
%(('られ/助動詞/られ', 'た/助動詞/た'), 1114)
%(('ん/語尾/ん', 'だ/助動詞/だ'), 854)
%((), 839)
%(('い/語尾/い', 'た/助動詞/た'), 761)
%(('わ/語尾/わ', 'れ/助動詞/れ', 'た/助動詞/た'), 715)
%(('で/助動詞/で', 'あ/動詞/あ', 'る/語尾/る'), 688)
%(('さ/語尾/さ', 'れ/助動詞/れ', 'た/助動詞/た'), 599)
%(('わ/語尾/わ', 'れ/助動詞/れ', 'る/語尾/る'), 595)
%(('れ/助動詞/れ', 'る/語尾/る'), 563)
%(('く/語尾/く',), 485)


% ('できる', 'ます'): 11, ('かもしれる', 'ます', 'ない'): 1, 
%('ない', 'た'): 29, ('ちゃう', 'ます', 'た'): 2, 
%('する', 'ます'): 23, ('られる', 'ます'): 4, ('できる',): 18, ('する', 'せる', 'た'): 12, ('たい', 'ない'): 1, ('する', 'ます', 'ない'): 4, ('せる', 'ない'): 1, ('らしい',): 2, ('出来る', 'ます', 'ない'): 1, ('かもしれる', 'ない'): 1, ('れる', 'ない', 'た'): 1, ('ない', 'だめ', 'だ'): 1, ('する', 'れる', 'そうだ'): 2, ('える',): 5, ('かねる', 'ない'): 1, ('れる', 'ようだ', 'なる', 'た'): 3, ('せる', 'ます', 'ない'): 2, ('続ける', 'た'): 1, ('られる', 'ようだ', 'なる', 'ます', 'た'): 1, ('できる', 'ます', 'た'): 7, ('する', 'せる'): 2, ('やすい', 'なる', 'た'): 1, ('くださる', 'ます'): 1, ('ようだ', 'なる', 'ます', 'た'): 2, ('える', 'ようだ', 'なる', 'た'): 1, ('する', 'がたい'): 1, ('する', 'れる', 'ます', 'た'): 6, ('ざるを得る', 'ます', 'ない'): 1, ('ようだ', 'なる', 'た'): 10, ('する', 'れる', 'ます'): 3, ('られる', 'ます', 'た'): 4, ('える', 'た'): 1, ('する', 'ます', 'う'): 1, ('する', 'ます', 'ない', 'でした'): 1, ('済み',): 1, ('ようだ',): 2, ('出す', 'た'): 1, ('出す',): 1, ('する', 'た', 'みたいだ'): 1, ('させる', 'た'): 2, ('きる', 'た'): 1, ('する', 'たい'): 2, ('続ける',): 2, ('ざるをえる', 'ます', 'ない'): 1, ('める', 'ます'): 6, ('せる', 'ます'): 1, ('ようだ', 'なる'): 7, ('られる', 'ようだ', 'なる', 'た'): 1, ('始める', 'た'): 2, ('ない', 'なる'): 2, ('かねる', 'ます'): 1, ('始める',): 2, ('する', 'れる', 'ない', 'た'): 1, ('える', 'ます'): 3, ('た', 'そうだ'): 4, ('れる', 'ます', 'ない'): 1, ('する', 'ようだ', 'なる'): 4, ('する', 'える', 'ない'): 1, ('ける', 'ます'): 3, ('できる', 'ない', 'た'): 1, ('する', 'ない', 'た'): 2, ('出来る', 'ます'): 1, ('れる', '始める'): 1, ('ます', 'う'): 4, ('する', '始める', 'た', 'そうだ'): 1, ('する', 'れる', 'た', 'ようだ'): 1, ('できる', 'まい'): 1, ('た', 'らしい'): 1, ('する', 'ようだ', 'なる', 'た'): 2, (' える', 'ます', 'ない'): 1, ('ける',): 2, ('られる', 'た', 'そうだ'): 1, ('られる', 'そうだ'): 1, ('れる', 'にくい'): 1, ('する', 'やすい'): 1, ('ける', 'た'): 1, ('する', 'れる', 'ます', 'ない'): 2, ('そ うだ',): 1, ('ない', 'た', 'ようだ'): 1, ('ます', 'たー'): 1, ('できる', 'た'): 3, ('べる',): 1, ('する', 'れる', 'ようだ', 'なる', 'ます', 'た'): 1, ('ない', 'ようだ'): 1, ('ける', 'ます', 'う'): 1, ('ちゃう', 'ます'): 1, ('する', 'そうだ'): 1, ('たい', 'なる'): 1, ('ようだ', 'なる', 'ます'): 1, ('たい', 'ない', 'た'): 1, ('やすい',): 3, ('かける',): 1, ('ない', 'なる', 'ようだ', 'なる', 'た'): 1, ('ない', 'そうだ'): 1, ('する', 'たい', 'ない', 'らしい'): 1, ('する', 'ちゃう', 'ます'): 1, ('た', 'ようだ'): 1, ('られる', 'たい', 'ない'): 1, ('える', 'ない'): 1, ('てる', 'ない'): 1, ('する', '易い', 'なる', 'た'): 1, ('れる', 'ない'): 1})


% nai (negation)





\subsection{Sesotho}

\cite{doke1967textbook}

We use the Demuth Corpus \cite{demuth1992acquisition} of child and child-directed speech, containing about 13K utterances with 500K morphemes.

The corpus has very extensive manual morphological segmentation and annotation; each verb form is segmented into morphemes, which are annotated for their function.

Sometimes morphemes fused, indicated with slash in the annotation.

Sesotho has composite forms consisting of an inflected auxiliary followed by an inflected verb.
Both verbs carry subject agreement.
While they are annotated as a unit in the Demuth corpus, they are treated as separate words in grammars \citep{doke1967textbook,guma1971outline,lombard1969handbook}.
We separated these, taking the main verb to start at its subject agreement prefix.
We only considered main verbs for the experiments here.

% TODO analysis for adult-oly portion

\paragraph{Verb Prefixes in Sesotho}

In the Demuth corpus, each morpheme is annotated; a two-letter key indicates the type of morpheme (e.g. subject agreement, TAM marker).
We classified morphemes by this annotation.
We considered morpheme types occurring at least 50 times in the corpus.

As in Japanese, morphemes show different forms depending on their environment, and the corpus contains some instances of fused neighboring morphemes that were not segmented further.

\begin{enumerate}


    \item SUBJECT Subject agreement (sm- 17K or sr- 193): person/number or noun class. also has variation depending on verb form (e.g. 1st sg is ke-, ka-, N- depending on form, Guma 1971, p. 162). See Lombard 1985, p. 104-106 for morphophonological rules.
    
    \cite[\textsection 395]{doke1967textbook} for sm-
    
    \item NEGATION Negation (ska, seka, sa, skaba, 362). sa used in Negative Dependent Present (Paroz 1946, p.31) and Perfect (ibid p. 80). E.g. Sec 13.38 p.172 in Guma 1971. -sa- and -se- Lombard 1985 p. 108, appear in different moods.
    
    \item TENSE/ASPECT Tense/aspect marker (t\^{} 13K) Guma 1971, p. 165-166. -a- Lombard 1985, p. 109.
    
    Some of the common markers in the corpus are t\^{}f for future and t\^{}pf for perfect.
    
    (e.g., ile- for remote past \cite[\textsection 177]{doke1967textbook})
    
    Also often fused with the object agreement marker.
    
    
    \item OBJECT Object agreement (om 6K) or reflexive (rf, 751). These are mutually exclusive \cite[p. 165]{guma1971handbook}
\end{enumerate}

Additionally FINITENESS (?) Conditional morpheme (\emph{ho}, `if', 314), comes before object agreement.

Parallel to Bybee hierarchy: subject agreement is farther away from the verb than TAM.


TODO ha- negation is a prefix coming before the subject agreement prefix (Guma 1971, p. 164).  ga- in Lomard 1985 p. 108.



Also some merging between tense and object markers. for the lemma-based version, have to construct a version where all of this merging is undone (e.g. including a slash, or t\^{}.om2s)


\paragraph{Verb Suffixes in Sesotho}

Again, we extracted morpheme types occurring at least 50 times.

\begin{enumerate}
    \item VALENCE:
    
    - applicative
    (ap -el- \cite[\textsection 310]{doke1967textbook} \cite[p. 109]{lombard1969handbook}, with morphophonological changes), 2K times
    
    \cite[\textsection 314-315]{doke1967textbook}: applicarive can be applied to other valence affixes
    
    - causative (c 1K), -isa (with morphophonological change) \cite[\textsection 325]{doke1967textbook}
    
    - neuter/stative (nt, 229), -eha, -ahala \cite[\textsection 307]{doke1967textbook}
    
    - reversive (rv, 214),  \cite[\textsection 345]{doke1967textbook} -- not really VALENCE, ex. bind -- loosen
    
    - rc reciprocal (-an 103 times)  \cite[\textsection 338]{doke1967textbook}
    
    \item VOICE: passive -iw- (1K)
    
    \cite[\textsection 300]{doke1967textbook} \cite[p. 114]{lombard1969handbook}
    
    Passive follows applicative: sho-el-oa die-APPL-PASS \cite[\textsection 324]{doke1967textbook}
    \item TENSE: tense (t\^{}, 3K) (e.g., -il...e for perfect), \cite[p. 167]{guma1971outline}, \cite[p. 116]{lombard1969handbook}.
    \item MOOD: Mood (m\^{}, 37K times)
    \item Interrogative (2K times) and relative (857 times) markers -ng
    
    interrogative: \cite[p. 168]{guma1971handbook}, \cite[\textsection 160, 320]{doke1967textbook}
    
    relative \cite[\textsection 271, 793]{doke1967textbook}
    
    \item     TODO cl COMPL -ets, -ell (66 times) have to find this in the grammars
\end{enumerate}

relation to Bybee hierarchy: valence closest, then voice, then tense, then mood.

- phonemic

- normalized to morphemes as annotated in the Demuth corpus

- normalized to morphemes, undoing merging


%('c', 'ap', 'm^'): 36, 
%('c', 'nt', 'm^'): 19, ('p', 't^', 'rl'): 12, 
%('p/', 'm^'): 1, ('wh', 'm^'): 1, ('rc', 'm^', 'rl'): 1, ('rv', 'c', 'm^'): 2, 
%('c', 'p', 'm^'): 32, ('c', 't^', 'p', 'm^'): 5, ('rc', 't^'): 17, ('c', 'ap', 'm^', 'wh'): 8, ('nt', 'm^', 'rl'): 10, ('rv', 'p', 'm^'): 3, ('ap', 't^'): 6, ('ap', 'cl', 'p', 'm^', 'wh'): 2, 
%('ap', 'm^', 'rl'): 39, ('ap', 'cl', 'm^'): 29, ('rv', 'ap', 'm^'): 4, ('c', 'm^', 'wh'): 11, ('ap', 'ap', 'p', 'm^', 'rl'): 1, ('rv', 't^', 'm^', 'rl'): 2, ('c', 't^'): 5, ('c/', 'p', 'm^'): 6, ('t^', 'rl'): 6, ('c', 'rc', 'm^'): 4, ('m.',): 1, ('rc', 'c', 'rc', 'm^'): 1, ('wh',): 1, ('nt', 't^'): 1, 

%('p', 'm^', 'rl'): 38, 

%('p', 't^', 'm^'): 1, ('t^', 'wh'): 2, ('rc', 'c', 'm^'): 3, ('ap', 'c', 'm^'): 3, ('c', 'ap', 't^', 'm^'): 3, ('cl', 'm^'): 23, ('rc', 'c', 'ap', 'm^'): 1, ('c', 'p', 't^'): 1, ('ap', 't^', 'm^', 'wh'): 2, ('p', 't^', 'wh'): 1, ('ap', 't^', 'm^'): 20, ('nt', 't^', 'm^', 'rl'): 2, ('ap', 't^', 'm^', 'rl'): 3, ('c', 'p', 'm^', 'rl'): 1, ('m^', 'lc'): 1, ('rv', 'p', 'm^', 'rl'): 1, ('ap', 'm^', 'lc'): 1, ('c', 'cl', 'm^'): 1, ('ap', 't^', 'p', 'm^'): 3, ('ap', 'ap', 'm^', 'wh'): 2, ('cp', 't^', 'm^'): 1, ('ap', 'ap', 't^', 'm^'): 3, ('c/', 'm^', 'wh'): 1, ('ap', 't^', 'rl'): 2, ('c', 'mi'): 1, ('rv', 'nt', 't^', 'm^'): 1, ('ps', 't^'): 3, ('t^', 'p', 'm^', 'wh'): 1, ('rc', 'm^', 'wh'): 1, ('ap', 'rc', 'm^'): 5, ('rv', 'nt', 'm^'): 2, ('ap', 'cl', 't^', 'm^'): 4, ('c', 't^', 'm^', 'rl'): 8, ('rc', 'p', 'm^'): 2, ('ap', 'p', 'rc', 'wh'): 1, ('ap', 'p', 'm^', 'wh'): 2, ('ap', 'cl', 'm^', 'wh'): 1, ('c', 't^', 'rl'): 1, ('cl', 'p', 'm^', 'wh'): 1, ('rv', 't^', 'p', 'm^', 'rl'): 1, ('ap', 'c/', 'm^'): 1, ('nt', 'ap', 'm^'): 1, ('rv', 't^', 'p', 'm^'): 1, ('cl', 't^', 'm^'): 3, ('rv', 'ap', 'm^', 'wh'): 1, ('c/', 'm^', 'rl'): 1, ('c', 'ap', 'p', 'm^'): 1, ('cl', 'ap', 'm^'): 1, ('ap', 'p', 'm^', 'rl'): 1, ('cl', 'p', 'm^'): 1})



