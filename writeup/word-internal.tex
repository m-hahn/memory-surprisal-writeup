- datasets

-- unimorph

-- bibles corpus?

-- celex


- permute phonemes

- permute syllables

- permute phonemes only within syllables

- permute morphemes 

\section{Morphology}

Bybee 1985:

verb-valence-voice-aspect-tense-mood-modality-subj.person-subj.number

can also directly quantify MI with verb stem (presence \& choice for each slot) as a sanity check


\subsection{Japanese}


\paragraph{Data Selection}

UD corpora 2.4 with freely available word forms

selected all chains VERB AUX...AUX. If a -te followed, added this.

We obtained X forms.

Describe format: form vs lemma.

Selected all morphemes occurring at least 50 times and annotated their meaning/function.

\paragraph{Verb Suffixes in Japanese}

We list the morphemes according to the order extracted according to the model.
While this ordering matches almost all observed forms, there are some orderings that do not fit into this pattern, see below.

Think we should exclude YOODA and NARU. By the same criteria, might want to exclude DEKIRU?

\begin{enumerate}
\item suru: derivational, turns Sino-Japanese nouns into verbs.
\item VALENCE: causative (-ase-). Hasegawa 142 (saseru, seru. 190)
\item VOICE: passive (-areru, -rareru). passive (Hasegawa 152) -areru, -rareru. (rareu, reru. 2000)
\item MODALITY: dekiru potential (180), -ta- (-tai, -taku-, -taka-) desiderative (tai. 85). but can also have masu after tai: ikitakunarimasu IKU-TAI-NARU-MASU `I want to go'. ORDER CONFLICT: mimashitai `I want to see'
\item POLITENESS (masu, mashi, mase. 600). comes between dekiru and tai % 見できます, 見ましたい
\item NEGATION: negation (-nai, -n-, -nakaC-. 630). also -mai for negative+volition.
\item yooda (180) EVIDENTIAL? yooda. but past -ta- can be after it: iruyoodatta IRU-YOODA-TA (Kaiser et al 9.5.6.1.1.1) vs misenakattayooda MISERU-NAI-TA-YOODA `it didn't seem' (GSD treebank). Maybe this should be thought of better as an SCONJ at least when including -ni, Vaccari p. 238-241.
\item naru (`become', 102). Maybe this should be excluded, not treated as an affix in the books.
\item TENSE/ASPECT: tense (-ta for past 4000, -yoo for hortative 92, ...)
\item TE (creates nonfintie form. 4000)
%\item -i (as in -tai, -nai, adjectives)
%\item EVIDENTIAL: tashii
\end{enumerate}


Some relatively frequent morphemes that are below the threshold

\begin{enumerate}
\item nakereba `if not'
\item tsudzukeru (continuative, 30)
\item yasui `easy to'
\item hajimeru (inchoative, 21)
\end{enumerate}


there's just too many verb forms, here limited to frequent ones we found in the corpus.

% volitional 加わろう kawawar-oo
% volitional polite 加わりましょう kawawar-i-mashy-oo
% volitional neg 加わるまい(+), 加わらない[よう/こと]にしよう kawawaru-mai, kawawara-nai-yoo-nishyoo
% volitional neg polite 加わりますまい, 加わらない[よう/こと]にしましょう  kawawari-masu-mai, kawawar-anai-yoo-nishimashyoo

%  verb-valence-voice-aspect-tense-mood-modality-person-number

% http://nihongo.monash.edu/cgi-bin/wwwjdic?1W%B2%F1%B5%C4%A4%CB%B2%C3%A4%EF%A4%EB_v5r
% passive negative
% 加わられない kuwawar-are-nai

% passive negative polite
% 加わられません kuwawar-are-mase-n

% causative polite
% 加わらせます kuwawar-ase-masu

% causative negative polite
% 加わらせません kuwawar-ase-mase-n
% 加わらしません kuwawar-ashi-mase-n

% da/de/na (lemmatized as da): copula (?). na after adjectives.


past negation: nakat-ta

mi-naka-tta did not see (Vaccari 153)

desiderative:

-tai, -takatta (past)

mi-taku-nai I do not wish to see (98)

mi-taku-naka-tta I didn't wish  to see

oyog-ase-ta `made swim'



honorific -masu

-s-are-naka-tta ?passive+negative.past?



% -u/-ru (non-past) -- not segmented off in UD?

% suru - reru - ta
% suru -masu - ta
% reru - ta
% suru - ta
% masu - ta
% rareru - ta
% masu - nai - deshita
% suru - reru
% suru - nai
% masu - nai
% reru - masu - ta
% dekiru - masu

% TODO in excluding -te-, have to deal with -de- as in 読んで yonde (Vaccari p. 109)

%(('し/動詞/し', 'た/助動詞/た'), 14389)
%(('る/語尾/る',), 13880)
%(('あ/動詞/あ', 'る/語尾/る'), 11481)
%(('っ/語尾/っ', 'た/助動詞/た'), 9378)
%(('さ/動詞/さ', 'れ/助動詞/れ', 'た/助動詞/た'), 7143)
%(('さ/動詞/さ', 'れ/助動詞/れ', 'る/語尾/る'), 1966)
%(('られ/助動詞/られ', 'る/語尾/る'), 1156)
%(('だっ/助動詞/だっ', 'た/助動詞/た'), 1151)
%(('られ/助動詞/られ', 'た/助動詞/た'), 1114)
%(('ん/語尾/ん', 'だ/助動詞/だ'), 854)
%((), 839)
%(('い/語尾/い', 'た/助動詞/た'), 761)
%(('わ/語尾/わ', 'れ/助動詞/れ', 'た/助動詞/た'), 715)
%(('で/助動詞/で', 'あ/動詞/あ', 'る/語尾/る'), 688)
%(('さ/語尾/さ', 'れ/助動詞/れ', 'た/助動詞/た'), 599)
%(('わ/語尾/わ', 'れ/助動詞/れ', 'る/語尾/る'), 595)
%(('れ/助動詞/れ', 'る/語尾/る'), 563)
%(('く/語尾/く',), 485)


% ('できる', 'ます'): 11, ('かもしれる', 'ます', 'ない'): 1, 
%('ない', 'た'): 29, ('ちゃう', 'ます', 'た'): 2, 
%('する', 'ます'): 23, ('られる', 'ます'): 4, ('できる',): 18, ('する', 'せる', 'た'): 12, ('たい', 'ない'): 1, ('する', 'ます', 'ない'): 4, ('せる', 'ない'): 1, ('らしい',): 2, ('出来る', 'ます', 'ない'): 1, ('かもしれる', 'ない'): 1, ('れる', 'ない', 'た'): 1, ('ない', 'だめ', 'だ'): 1, ('する', 'れる', 'そうだ'): 2, ('える',): 5, ('かねる', 'ない'): 1, ('れる', 'ようだ', 'なる', 'た'): 3, ('せる', 'ます', 'ない'): 2, ('続ける', 'た'): 1, ('られる', 'ようだ', 'なる', 'ます', 'た'): 1, ('できる', 'ます', 'た'): 7, ('する', 'せる'): 2, ('やすい', 'なる', 'た'): 1, ('くださる', 'ます'): 1, ('ようだ', 'なる', 'ます', 'た'): 2, ('える', 'ようだ', 'なる', 'た'): 1, ('する', 'がたい'): 1, ('する', 'れる', 'ます', 'た'): 6, ('ざるを得る', 'ます', 'ない'): 1, ('ようだ', 'なる', 'た'): 10, ('する', 'れる', 'ます'): 3, ('られる', 'ます', 'た'): 4, ('える', 'た'): 1, ('する', 'ます', 'う'): 1, ('する', 'ます', 'ない', 'でした'): 1, ('済み',): 1, ('ようだ',): 2, ('出す', 'た'): 1, ('出す',): 1, ('する', 'た', 'みたいだ'): 1, ('させる', 'た'): 2, ('きる', 'た'): 1, ('する', 'たい'): 2, ('続ける',): 2, ('ざるをえる', 'ます', 'ない'): 1, ('める', 'ます'): 6, ('せる', 'ます'): 1, ('ようだ', 'なる'): 7, ('られる', 'ようだ', 'なる', 'た'): 1, ('始める', 'た'): 2, ('ない', 'なる'): 2, ('かねる', 'ます'): 1, ('始める',): 2, ('する', 'れる', 'ない', 'た'): 1, ('える', 'ます'): 3, ('た', 'そうだ'): 4, ('れる', 'ます', 'ない'): 1, ('する', 'ようだ', 'なる'): 4, ('する', 'える', 'ない'): 1, ('ける', 'ます'): 3, ('できる', 'ない', 'た'): 1, ('する', 'ない', 'た'): 2, ('出来る', 'ます'): 1, ('れる', '始める'): 1, ('ます', 'う'): 4, ('する', '始める', 'た', 'そうだ'): 1, ('する', 'れる', 'た', 'ようだ'): 1, ('できる', 'まい'): 1, ('た', 'らしい'): 1, ('する', 'ようだ', 'なる', 'た'): 2, (' える', 'ます', 'ない'): 1, ('ける',): 2, ('られる', 'た', 'そうだ'): 1, ('られる', 'そうだ'): 1, ('れる', 'にくい'): 1, ('する', 'やすい'): 1, ('ける', 'た'): 1, ('する', 'れる', 'ます', 'ない'): 2, ('そ うだ',): 1, ('ない', 'た', 'ようだ'): 1, ('ます', 'たー'): 1, ('できる', 'た'): 3, ('べる',): 1, ('する', 'れる', 'ようだ', 'なる', 'ます', 'た'): 1, ('ない', 'ようだ'): 1, ('ける', 'ます', 'う'): 1, ('ちゃう', 'ます'): 1, ('する', 'そうだ'): 1, ('たい', 'なる'): 1, ('ようだ', 'なる', 'ます'): 1, ('たい', 'ない', 'た'): 1, ('やすい',): 3, ('かける',): 1, ('ない', 'なる', 'ようだ', 'なる', 'た'): 1, ('ない', 'そうだ'): 1, ('する', 'たい', 'ない', 'らしい'): 1, ('する', 'ちゃう', 'ます'): 1, ('た', 'ようだ'): 1, ('られる', 'たい', 'ない'): 1, ('える', 'ない'): 1, ('てる', 'ない'): 1, ('する', '易い', 'なる', 'た'): 1, ('れる', 'ない'): 1})


% nai (negation)





\subsection{Sesotho}



\paragraph{Verb Prefixes in Sesotho}

Most common:
\begin{enumerate}
    \item Subject agreement (sm or sr): person/number or noun class. also has variation depending on verb form (e.g. 1st sg is ke-, ka-, N- depending on form, Guma 1971, p. 162). See Lombard 1985, p. 104-106 for morphophonological rules.
    \item Tense/aspect marker (t{\^\ }) Guma 1971, p. 165-166. -a- Lombard 1985, p. 109.
    \item Object agreement (om)
\end{enumerate}


TODO ha- negation is a prefix coming before the subject agreement prefix (Guma 1971, p. 164).  ga- in Lomard 1985 p. 108.

Also some merging between tense and object markers. for the lemma-based version, have to construct a version where all of this merging is undone (e.g. including a slash, or \begin{verbatim}t^.om2s\end{verbatim})

Other morphemes
\begin{enumerate}
    \item SUBJECT: Subject agreement
    \item (TODO) Copula (cp)
    \item NEGATION Negation (ska, seka, sa, skaba). sa used in Negative Dependent Present (Paroz 1946, p.31) and Perfect (ibid p. 80). E.g. Sec 13.38 p.172 in Guma 1971. -sa- and -se- Lombard 1985 p. 108, appear in different moods.
    \item TAM Tense/aspect marker (e.g., ile- for remote past)
    \item (TODO) conditional morpheme (\emph{ho}, `if')
    \item OBJECT Object agreement (optional) or Reflexive (rf, i-) (exclusive: Guma 1971, p. 165)
\end{enumerate}


\paragraph{Verb Suffixes in Sesotho}


\begin{enumerate}
    \item VALENCE: applicative (ap -el- Lombard 1985 p. 109, with morphophonological changes), causative (c), neuter/stative (nt), reversive (rv), rc reciprocal
    \item VOICE: passive (p, -iw- Lombard 1985, p. 114)
    \item TENSE: tense (e.g., -il...e for perfect), Guma 1971 p. 167, Lombard 1985 p. 116.
    \item MOOD: Mood
    \item (TODO) wh marker, relative marker (rl, -ng Guma 1971, p. 168)
\end{enumerate}



%('c', 'ap', 'm^'): 36, 
%('c', 'nt', 'm^'): 19, ('p', 't^', 'rl'): 12, 
%('p/', 'm^'): 1, ('wh', 'm^'): 1, ('rc', 'm^', 'rl'): 1, ('rv', 'c', 'm^'): 2, 
%('c', 'p', 'm^'): 32, ('c', 't^', 'p', 'm^'): 5, ('rc', 't^'): 17, ('c', 'ap', 'm^', 'wh'): 8, ('nt', 'm^', 'rl'): 10, ('rv', 'p', 'm^'): 3, ('ap', 't^'): 6, ('ap', 'cl', 'p', 'm^', 'wh'): 2, 
%('ap', 'm^', 'rl'): 39, ('ap', 'cl', 'm^'): 29, ('rv', 'ap', 'm^'): 4, ('c', 'm^', 'wh'): 11, ('ap', 'ap', 'p', 'm^', 'rl'): 1, ('rv', 't^', 'm^', 'rl'): 2, ('c', 't^'): 5, ('c/', 'p', 'm^'): 6, ('t^', 'rl'): 6, ('c', 'rc', 'm^'): 4, ('m.',): 1, ('rc', 'c', 'rc', 'm^'): 1, ('wh',): 1, ('nt', 't^'): 1, 

%('p', 'm^', 'rl'): 38, 

%('p', 't^', 'm^'): 1, ('t^', 'wh'): 2, ('rc', 'c', 'm^'): 3, ('ap', 'c', 'm^'): 3, ('c', 'ap', 't^', 'm^'): 3, ('cl', 'm^'): 23, ('rc', 'c', 'ap', 'm^'): 1, ('c', 'p', 't^'): 1, ('ap', 't^', 'm^', 'wh'): 2, ('p', 't^', 'wh'): 1, ('ap', 't^', 'm^'): 20, ('nt', 't^', 'm^', 'rl'): 2, ('ap', 't^', 'm^', 'rl'): 3, ('c', 'p', 'm^', 'rl'): 1, ('m^', 'lc'): 1, ('rv', 'p', 'm^', 'rl'): 1, ('ap', 'm^', 'lc'): 1, ('c', 'cl', 'm^'): 1, ('ap', 't^', 'p', 'm^'): 3, ('ap', 'ap', 'm^', 'wh'): 2, ('cp', 't^', 'm^'): 1, ('ap', 'ap', 't^', 'm^'): 3, ('c/', 'm^', 'wh'): 1, ('ap', 't^', 'rl'): 2, ('c', 'mi'): 1, ('rv', 'nt', 't^', 'm^'): 1, ('ps', 't^'): 3, ('t^', 'p', 'm^', 'wh'): 1, ('rc', 'm^', 'wh'): 1, ('ap', 'rc', 'm^'): 5, ('rv', 'nt', 'm^'): 2, ('ap', 'cl', 't^', 'm^'): 4, ('c', 't^', 'm^', 'rl'): 8, ('rc', 'p', 'm^'): 2, ('ap', 'p', 'rc', 'wh'): 1, ('ap', 'p', 'm^', 'wh'): 2, ('ap', 'cl', 'm^', 'wh'): 1, ('c', 't^', 'rl'): 1, ('cl', 'p', 'm^', 'wh'): 1, ('rv', 't^', 'p', 'm^', 'rl'): 1, ('ap', 'c/', 'm^'): 1, ('nt', 'ap', 'm^'): 1, ('rv', 't^', 'p', 'm^'): 1, ('cl', 't^', 'm^'): 3, ('rv', 'ap', 'm^', 'wh'): 1, ('c/', 'm^', 'rl'): 1, ('c', 'ap', 'p', 'm^'): 1, ('cl', 'ap', 'm^'): 1, ('ap', 'p', 'm^', 'rl'): 1, ('cl', 'p', 'm^'): 1})



